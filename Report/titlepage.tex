\pdfbookmark[0]{Titelblad}{label:titlepage_da}
\aautitlepage{%
  \danishprojectinfo{
    PsyLog: Søvn og Aktivitetsmoduler for Personer med Affektive Lidelser %title
  }{%
    Mobile Systemer %theme
  }{%
    P8, Forårssemestret 2015 %project period
  }{%
    SW808F15 % project group
  }{%
    %list of group members
    Søren Skibsted Als\\
    Lars Andersen\\
    Lasse Vang Gravesen\\
    Mathias Winde Pedersen
    
  }{%
    %list of supervisors
    Ivan Aaen
  }{%
    1 % number of printed copies
  }{%
    27. maj 2015 % date of completion
  }%
}{%department and address
  \textbf{Information og kommunikations teknologi}\\
  Selma Lagerlöfsvej 300\\
  Aalborg Universitet\\
  \href{http://www.aau.dk}{http://www.aau.dk}
}{
Patients suffering from mood disorder typically experience change in their sleep and physical activity patterns when they start a mania or depression period.

We research how to develop sleep and physical activity estimation modules that utilises the PsyLog platform.

The modules are intended to be used in a self-help treatment setting.

Based on research regarding sleep estimation algorithms, a proof of concept is presented.
The proof of concept can be seen as an initial solution but should be developed further before being used by the patients.

Additionally, we give a short description of the development of a pedometer to track physical activity to be used in a self-help treatment setting.


%Typisk for patienter med affektive lidelser, er at ved start på en mani- eller depressions-periode ændrer deres søvnvaner og fysisk aktivitet sig fra normalen.

%Der undersøges muligheden for at udvikle søvn og fysisk aktivitets estimeringsmoduler der benytter sig af PsyLog platformen.

%Modulerne er tiltænkt til at bruges i en behandlingssammenhæng for patienter med affektive lidelser, og er tænkt som en hjælp til selvhjælp.

%Ud fra et forarbejde til undersøgelse af søvnestimeringsalgoritmer, bliver en proof of concept løsning præsenteret.
%Løsningen er et første bud på en løsning, men kan med fordel arbejdes videre på.

%Derudover, følger der en kort beskrivelse af estimering af fysisk aktivitet i form af en skridttæller, og af hvordan et sådant modul kan være nyttig i en behandlingssammenhæng.



}
