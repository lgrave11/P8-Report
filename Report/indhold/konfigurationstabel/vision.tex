For at have styr på hvilken retning et projektet skal gå, er et fælles vision en god ting at have.
Dette kapitel forklare derfor det fælles vision for dette projekt så alle valg følger dette vision.
Det fælles vision består af to under visioner, som er søvn og fælles aktivitet da disse er de to store dele af projektet.

Kigger man på det store vision for projektet så er det at lave et redskab til patienter med affektive lidelser som skal kunne give dem et overblik over deres sindstilstand.
For at kunne løse dette vision kigges der på hvordan man med en smartphone kan se indikatorer for at man er i et tidlig stadie.
Til at begrænse dette vision vil der kun kigges på nogle få mulige indikatorer hvilket er som nævn før søvn og fysisk aktivitet.
Hvilket derfor giver det en fælles vision om at udvikle en applikation til smartphones der ved hjælp af søvn og fysisk aktivitet giver patienter med affektive lidelser et overblik over deres sindstilstand.
For at løse dette vision vil der herefter blive skrevet et vision for søvn og fysisk aktivitet, da disse er to forskellige dele af projektet.

