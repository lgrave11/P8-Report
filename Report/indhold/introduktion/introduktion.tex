Idag er der mange danskere som lider af psykiske lidelser hvilket tydeligt kan uddrages fra bruget af antidepressiv medicin idet at 1 af 12 danskere i 2011 brugte det \citep{misc:forbrugAntidepressiva}. 

På samme tid har næsten 80\% af danskere adgang til smartphones \citep{misc:dstElektronik}.

Idet at der er så mange mennesker som lider af disse lidelser, giver det mening at forsøge innovere i det område og lave et mobilt system som kan hjælpe brugerne.

Dette ligger som baggrund for ´PsyLog' platformen, REF TIL FÆLLES RAPPORT, hvor et mobilt system til indsamling af sensor og brugs data er blevet gjort muligt, men der er på samme tid ikke blevet lavet noget som aktuelt kan bruges af personer med psykiske lidelser.

I kombination med ´PsyLog' platformen og idéen at ved mange psykiske lidelser er der opførsels mønstre som måske kan opdages tidligt og derved informere brugeren og på denne måde hjælpe dem. 
Det vil sige at der kan være indikatorer på om personer er begyndt af få det værre, som f.eks. skizofrene oplever nedsat social interaktion KILDE HER.

Vores fokus er som allerede sagt i fælles rapporten på personer med unipolar og bipolar depression. 
For personer med disse lidelser er der forskellige indikatorer, som f.eks. nedsat fysisk aktivitet og social aktivitet, søvn problemer, indre uro, dårlig appetit, humør, man skal tvinge sig selv til at gøre alting og så videre.
Af disse kan fysisk og social aktivitets niveau søvn måles ved hjælp af data som kan optages fra en smartphone. 
Her vælger vi at fokusere primært på søvn da det gentagne gange er blevet sagt fra psyloger, psykiatere og patienter(KILDER) at søvn problemer er et af de mest sete symptomer på unipolar og bipolar depression, på samme tid vil vi også prøve at fokusere på fysisk aktivitets niveau da det kan simplificeres og måles meget let.

På baggrund af dette ville det være idéelt at udvikle moduler som kan hjælpe disse personer idet at hvis de kan se hvor meget de sover eller hvor aktive de er kan de få et overblik over deres situation og tage beslutninger om hvad de skal gøre. 
Dette vil inkludere udvikling af moduler som enten gemmer ny data fra en sensor eller analyserer eksisterende data.

Denne del af rapporten vil dokumentere overvejelser, beslutninger og implementeringen af dette.