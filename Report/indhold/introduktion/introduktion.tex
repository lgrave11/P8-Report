Idag er der mange danskere som lider af psykiske lidelser hvilket tydeligt kan ses fra bruget af antidepressiv medicin idet at 1 af 12 danskere i 2011 brugte det \citep{misc:forbrugAntidepressiva}. 

Disse personer har selvfølgelig helbreds veje til at komme under behandling, så som terapi behandling eller medicinsk behandling.
Men som patienter ved et fokusgruppe møde sagde, REF, ville ethvert værktøj som kan gøre deres liv nemmere eller bedre være godt.
Der er mange åbne veje for innovation i dette område, men noget af det mest relevante og interessante er at se på hvordan man kan bruge smartphones til at hjælpe disse personer.
Dette er klart da næsten 80\% af danskere har adgang til smartphones \citep{misc:dstElektronik}, og disse smartphones er et kraftfuldt værktøj som kan bruges på mange forskellige måder og mange af disse uudforskede.

Dette ligger som baggrund for den modulære fælles platform(eller `PsyLog'), REF TIL FÆLLES RAPPORT, hvor et mobilt system til indsamling af sensor og brugs data er blevet gjort mulig samt en platform som gør det muligt at lave udregninger og visninger af disse.

Ved mange psykiske lidelser ser man tit at der er opførsels mønstre som signalerer selve sygdommen, idet at visse sygdomme har symptomer som materaliserer sig som ændring i opførsel.
Et eksempel på dette kunne f.eks. hvad man ser med scizofrene som har nedsat social interaktion \citep{misc:negativeSymptomsSchizo}. 

I kombination med den fælles platform og faktaen at sygdomme har symptomer som materaliserer sig som ændring i opførsel og at dette muligvis kan opdages tidligt gennem brug af sensor data og derved bruges til at informere brugeren. 

Vores fokus er som allerede sagt i fælles rapporten på personer med unipolar og bipolar depression. 
For personer med disse lidelser er der forskellige indikatorer, som f.eks. nedsat fysisk aktivitet og social aktivitet, søvn problemer, indre uro, dårlig appetit, humør, man skal tvinge sig selv til at gøre alting og så videre. % ref til fokusgruppe møde
Af disse kan fysisk og social aktivitets niveau søvn måles ved hjælp af data som kan optages fra en smartphone. 
Her vælger vi at fokusere primært på søvn da det gentagne gange er blevet sagt fra psyloger, psykiatere og patienter(ref til det de sagde) at søvn problemer er et af de mest sete symptomer når det kommer til personer med unipolar og bipolar depression, på samme tid vil det også blive forsøgt at måle på fysisk aktivitets niveau da det kan simplificeres og måles meget let.

På baggrund af dette giver det mening at udvikle moduler som kan hjælpe disse personer idet at hvis de kan se hvor meget de sover eller hvor aktive de er giver det dem et overblik og dette vil så hjælpe dem med at tage beslutninger om hvad de skal gøre. 
Dette vil inkludere udvikling af moduler som enten gemmer ny data fra en sensor eller analyserer eksisterende data.