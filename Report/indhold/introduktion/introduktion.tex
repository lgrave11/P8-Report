I dagens Danmark er der en del, som lider af psykiske lidelser, hvilket tydeligt kan ses da 1 ud af 12 danskere i 2011 brugte antidepressiv medicin \citep{misc:forbrugAntidepressiva}. 

Disse personer har mulighed for at komme under behandling, hvilket kan være terapi eller medicinsk behandling.
Som patienterne til fokusgruppeinterviewet sagde, \citep[Kapitel 1, Sektion 5]{misc:faellesrapp}, vil ethvert værktøj, som kan gøre deres liv nemmere være godt.
Der er mange mulige måder at lave noget nyt i dette område, men noget af det mest relevante og interessante er, hvordan man kan bruge smartphones til at hjælpe patienter med psykiske lidelser.
Dette er en åbenlys mulighed da næsten 80\% af alle danskere har adgang til smartphones \citep{misc:dstElektronik}, og disse smartphones er et kraftfuldt værktøj, som kan bruges på mange forskellige måder, og mange af disse er ikke udforsket.

Dette ligger som baggrund for den fælles platform \citep{misc:faellesrapp}, hvor et mobilt system til indsamling af sensor og brugs data er blevet implementeret. 
Endvidere er det en platform, som gør det muligt at lave udregninger og visninger af sensor og brugs data.
Idéen ved det overordnede system er, at det skal fungere modulært, idet at der kan være forskellige moduler, som udfører opgaver såsom indsamling eller analyse af data.

Ved mange psykiske lidelser ser man tit at der er adfærdsmønstre, som signalerer selve sygdommen, idet at visse patienter har symptomer der materialiserer sig som ændring i adfærd.
Et eksempel på dette kunne være skizofreni, hvor social interaktion er nedsat \citep{misc:negativeSymptomsSchizo}.
Idéen er så at hvis man kan observere disse adfærdsændringer, ved brug af sensor data fra telefonen, kan man informere brugeren og give dem et bedre overblik over deres psykiske helbred.
Denne idé vil muligvis kunne fungere som et 'early warning' system, der kan opdage udbrud af sygdommen på et tidligt stadie.

Vores fokus er på personer med unipolar og bipolar lidelse. 
For personer med disse lidelser er der forskellige indikatorer, der er afhængig af individet, som er nedsat fysisk aktivitet, nedsat social aktivitet, søvn problemer, indre uro, dårlig appetit, ændring i humør og så videre \citep{misc:faellesrapp}.
Af disse kan eksempelvis fysisk, social aktivitets niveau og søvn måles ved hjælp af data, der kan optages fra en smartphone.

Forskning viser at søvn har en væsentlig indflydelse på ens sindstilstand, dette gælder i høj grad for bipolare patienter \citep{CPSP:CPSP1164}.
Endvidere, har vi også hørt gentagne gange fra psykologer, psykiatere og patienter, at søvnproblemer er meget fremtrædende for personer, der har enten unipolar eller bipolar lidelse \citep[Kapitel 2, Sektion 3,4,5]{misc:faellesrapp}.
Hvis man kan vise personer med en af disse lidelser, at de har søvn problemer på et tidligt stadie vil dette være fordelagtigt, da det så kan give mulighed for at gribe ind så snart der er signaler på at en depressions- eller maniperiode er under opsejling.

På samme tid er ændring i fysisk aktivitet også meget vigtigt, da man også ser dette ofte hos personer med unipolar eller bipolar lidelse, hvilket er baseret på hvad der blev sagt ved mødet med Jørgen Aagaard, se \citep[Kapitel 1, Sektion 4]{misc:faellesrapp}. 
Fysisk aktivitet kan simplificeres meget og måles relativt nemt ved at ændre perspektivet på hvad udgør fysisk aktivitet, f.eks. ved at se på hvor mange skridt man tager på en dag.

På baggrund af dette giver det mening at kigge på både hvor meget personer sover, men også på hvor fysisk aktive de er.
Overordnet set kan disse to parametre give et ganske godt indblik i en persons adfærd.
Grunden til dette er fordi det dækker både vågen og søvn perioder, ydermere giver disse mulighed for at opdage ændringer i adfærd, som ifølge Jørgen Aagaard \citep[Kapitel 1, Sektion 4]{misc:faellesrapp}, er det største signal på om en depressions- eller maniperiode er under opsejling. 

Derfor vil disse to undersøges samt forsøges at blive udviklet i konteksten af `PsyLog'.