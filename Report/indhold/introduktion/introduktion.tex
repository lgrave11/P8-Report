Idag er der en del danskere som lider af psykiske lidelser hvilket tydeligt kan ses fra bruget af antidepressiv medicin idet at 1 af 12 danskere i 2011 brugte det \citep{misc:forbrugAntidepressiva}. 

Disse personer har selvfølgelig veje til at komme under behandling, så som terapi eller medicinsk behandling.
Men som patienter ved et fokusgruppe møde sagde, REF, ville ethvert værktøj som kan gøre deres liv nemmere være godt.
Der er mange åbne veje for innovation i dette område, men noget af det mest relevante og interessante er at se på hvordan man kan bruge smartphones til at hjælpe disse personer.
Dette er en åbenlys vej da næsten 80\% af danskere har adgang til smartphones \citep{misc:dstElektronik}, og disse smartphones er et kraftfuldt værktøj som kan bruges på mange forskellige måder og mange af disse uudforskede.

Dette ligger som baggrund for den fælles platform(eller `PsyLog'), REF TIL FÆLLES RAPPORT, hvor et mobilt system til indsamling af sensor og brugs data er blevet implementeret samt en platform som gør det muligt at lave udregninger og visninger af disse, hvor det er tiltænkt at det overordnede system skal fungere modulært idet at der kan være forskellige moduler som udfører opgaver såsom indsamling eller analyse af data.

Ved mange psykiske lidelser ser man tit at der er opførsels mønstre som signalerer selve sygdommen, idet at visse sygdomme har symptomer som materaliserer sig som ændring i opførsel.
Et eksempel på dette kunne f.eks. hvad man ser med skizofrene som har nedsat social interaktion \citep{misc:negativeSymptomsSchizo}.
Idéen er så at hvis man kan observere disse opførsels mønstre ved brug af sensor data på den fælles platform kan man informere brugeren og give dem et bedre overblik over deres psykiske helbred og muligvis fungere som et 'early warning' system der kan få dem til at søge hjælp før symptomerne bliver meget fremtrædne.

Vores fokus er som allerede sagt i fælles rapporten på personer med unipolar og bipolar depression. 
For personer med disse lidelser er der forskellige indikatorer, som f.eks. nedsat fysisk aktivitet og social aktivitet, søvn problemer, indre uro, dårlig appetit, humør, man skal tvinge sig selv til at gøre alting og så videre. % ref til fokusgruppe møde
Af disse kan fysisk og social aktivitets niveau samt søvn måles ved hjælp af data som kan optages fra en smartphone.

Søvn er, baseret på hvad vi gentagne gange har hørt fra psykologer, psykiatere og patienter(REF), et af de mest sete symptomer når det kommer til personer med unipolar eller bipolar depression. 
Hvis man kan vise personer med en af disse lidelser at deres søvn problemer måske er endnu værre end de vidste ville det give mening at undersøge søvn. 
På samme tid er fysisk aktivitet også meget vigtigt, da man igen ser dette meget ofte hos personer med unipolar eller bipolar depression.
Fysisk aktivitet kan så også simplificeres meget og derved måles relativt nemt ved at ændre perspektivet på hvad udgør fysisk aktivitet, f.eks. ved at se på hvor mange skridt man tager på en dag.
Derfor vil dette også undersøges.

På baggrund af dette giver det mening at udvikle moduler som kan hjælpe disse personer idet at hvis de kan se hvor meget de sover eller hvor aktive de er giver det dem et overblik og dette vil så hjælpe dem med at tage beslutninger om hvad de skal gøre. 
Dette vil inkludere udvikling af moduler som enten gemmer ny data fra en sensor eller analyserer eksisterende data.