I dagens Danmark er der en del som lider af psykiske lidelser hvilket tydeligt kan ses fra brugen af antidepressiv medicin idet at 1 ud af 12 danskere i 2011 brugte det \citep{misc:forbrugAntidepressiva}. 

Disse personer har mulighed for at komme under behandling, hvilket kan være terapi eller medicinsk behandling.
Men som patienter ved et fokusgruppe møde sagde, se fælles rapport for referat\citep{misc:faellesrapp}, ville ethvert værktøj som kan gøre deres liv nemmere være godt \als{skal måske rykkes ind, ethvert værktøj... that is}.
Der er mange mulige måder at lave noget innovativt i dette område, men noget af det mest relevante og interessante er hvordan man kan bruge smartphones til at hjælpe dem som lider af psykiske lidelser.
Dette er en åbenlys mulighed da næsten 80\% af danskere har adgang til smartphones \citep{misc:dstElektronik}, og disse smartphones er et kraftfuldt værktøj som kan bruges på mange forskellige måder og mange af disse er ikke udforskede.

Dette ligger som baggrund for den fælles platform(eller `PsyLog')\citep{misc:faellesrapp}, hvor et mobilt system til indsamling af sensor og brugs data er blevet implementeret samt en platform som gør det muligt at lave udregninger og visninger af disse, hvor det er tiltænkt at det overordnede system skal fungere modulært idet at der kan være forskellige moduler som udfører opgaver såsom indsamling eller analyse af data.

Ved mange psykiske lidelser ser man tit at der er opførsels mønstre som signalerer selve sygdommen, idet at visse patienter har symptomer der materialiserer sig som ændring i adfærd.
Et eksempel på dette kunne være skizofreni, hvor social interaktion er nedsat \citep{misc:negativeSymptomsSchizo}.
Idéen er så at hvis man kan observere disse adfærds ændringer ved brug af sensor data på den fælles platform, kan man informere brugeren og give dem et bedre overblik over deres psykiske helbred, og muligvis fungere som et 'early warning' system der kan opdage udbrud af sygdommen på et tidligt stadie.

Vores fokus er på personer med unipolar og bipolar depression. 
For personer med disse lidelser er der forskellige indikatorer, som f.eks. nedsat fysisk aktivitet, nedsat social aktivitet, søvn problemer, indre uro, dårlig appetit, ændring i humør og så videre\citep{misc:faellesrapp}.
Af disse kan fysisk og social aktivitets niveau samt søvn måles ved hjælp af data der kan optages fra en smartphone.

Søvnmangel og for meget søvn er et af de mest sete symptomer når det kommer til personer med unipolar eller bipolar depression, baseret på hvad vi gentagne gange har hørt fra psykologer, psykiatere og patienter \citep{misc:jorgen-aagaard, misc:janne-rasmussen, misc:faellesrapp}.
Hvis man kan vise personer med en af disse lidelser at de har søvn problemer på et tidligt stadie vil dette være fordelagtigt, da det så kan give mulighed for at gribe ind før et udbrud manifesterer sig.\ivan{lidt uklart}\lars{forsøgt rettet}
På samme tid er ændring i fysisk aktivitet også meget vigtigt, da man også ser dette ofte hos personer med unipolar eller bipolar lidelse, hvilket også er baseret på (REF)\lars{find den kilde}.
Fysisk aktivitet kan simplificeres meget og måles relativt nemt ved at ændre perspektivet på hvad udgør fysisk aktivitet, f.eks. ved at se på hvor mange skridt man tager på en dag.
På baggrund af dette giver det mening at kigge på både hvor meget en personer sover, men også på hvor fysisk aktive de er.
Set på overordnet kan disse to parammetre give et ganske godt indblik i en persons adfærd, og gennem disse give mulighed for at opdage ændringer i adfærd, som ifølge \citet{misc:jorgen-aagaard} er det største signal på om en depressions- eller maniperiode er under opsejling. 
\ivan{uddyb gerne}\lars{forsøgt fixet}