Afslutningsvis kan vi konkludere at vi har fået et indblik i løsninger til søvn og fysisk aktivitets estimering.
Dog må vi erkende at det er et stort emne der kræver mere arbejde.
Den nuværende løsning skal ses som et spadestik i et længere udviklingsforløb, men skaber et grundlag for videre arbejde med emnet.
Derudover skal de omtalte moduler ses som enkelte moduler i den større PsyLog software pakke.
I den sammenhæng kan andre moduler udvikles til at supplere den udviklede analyse, men kan også være andre moduler der kan estimere i andre henseender, eksempelvis søvnkvalitet.


Men selvom der havde været mere tid til arbejdet anses det at dette domæne altid kan videreudvikles.
Med dette tænkes at der i fremtiden med stor sandsynlighed vil komme nye algoritmer til estimeringen samt nyt teknologisk udstyr der kan give nye datakilder man bør betragte.
Problemets natur er at man aldrig kan finde en færdig løsning, hvorfor platformen da også er udviklet med dette i tankerne.
Dermed ikke sagt at en tilstrækkelig løsning ikke kan findes, men blot at løsningen altid ville kunne forbedres.

Afslutningsvis kan vi konkludere at et proof of concept til estimering af søvn og fysisk aktivitet er udviklet, men disse løsninger er ikke færdige og bør forbedres ved videre arbejde.


\ivan{Vurdering af metoder til udvikling af målemetoder, til konceptudvikling. Styrker og svagheder.}
\ivan{Vurdering af perspektiverne for denne slags teknologier inden for psykiatrien.}
\ivan{Vurdering af perspektiver uden for psykiatrien (almen brug)}

% Stort emne
% Fysisk aktivitet og søvn analyse kan lade sig gøre
% kræver mere arbejde
% Blive del af større software pakke (PsyLog)
% Ikke færdigt - men danner et grundlag for en mere komplet løsning
% Problemets natur gør at man aldrig er færdig - men har også været en tankegang der er fulgt i udviklingen (fleksibilitet - modularitet)
