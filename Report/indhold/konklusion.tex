Afslutningsvis kan vi konkludere at vi har fået et indblik i løsninger til søvn- og fysisk aktivitetsestimering.
Dog må vi erkende at det er et stort emne der kræver mere arbejde.
Den nuværende løsning skal ses som et spadestik i et længere udviklingsforløb, men skaber et grundlag for videre arbejde med emnet.
Derudover skal de omtalte moduler ses som enkelte moduler i den større PsyLog software pakke.
I den sammenhæng kan andre moduler udvikles til at supplere den udviklede analyse, hvilket kan være ved at analysere søvn, men også andre henseender som eksempelvis søvnkvalitet.

Det tænkes, at der i fremtiden med stor sandsynlighed vil komme nye metoder til estimeringen samt nyt teknologisk udstyr, der kan give nye datakilder man bør betragte.
Problemets natur er at man aldrig kan finde en færdig løsning, hvorfor platformen da også er udviklet med dette i tankerne.
Dermed ikke sagt at en tilstrækkelig løsning ikke kan findes, men blot at løsningen altid ville kunne forbedres.

Vi kan konkludere at et proof of concept til estimering af søvn og fysisk aktivitet er udviklet, men disse løsninger er ikke færdige og bør forbedres ved videre arbejde.

Vores proof of concept løsning gør det dog nemmere at arbejde videre på søvnestimering.
Vi har opsat en primitiv måde at estimere søvn og fysisk aktivitet på, men det bygger ud fra en platform der er fleksibel og modular.
Denne tankegang følges også ved vores estimeringsmoduler, hvorfor det vurderes at dele af estimeringsmodulerne nemt kan udskiftes.
Eksempelvis kan den omtalte logistiske funktion nemt blive udskiftet til en anden metode.
Endvidere er vores vurdering af stilstand op til debat. 
Vores vurdering af stilstand er lagt i en separat metode og kan derfor også nemt udskiftes. 
Til sidst, som nævnt, kan de omtalte parametre også justeres, til at sikre en højere modenhed af estimeringsmodulerne.

%Vurdering af perspektiverne for denne slags teknologier inden for psykiatrien.


Det vurderes at når vores proof of concepts har nået en tilpas modenhed, hvilket er afgjort med den beskrevne verificering, kan det tages i brug i psykiatrien, til at give patienten indsigt i deres søvn og fysisk aktivitet. 
Ud fra samtaler med patienterne beskrevet i \citet{misc:faellesrapp}, fik vi det indtryk at de var åbne for brugen af et sådant system, i deres øjne var enhver ny metode velkommen.
Der forestilles at patienter bruger det udviklede system til at få overblik over deres fysiske aktivitet og søvn, og ved indikation på forværring af dette vil de kunne søge hjælp. 
Dette er knyttet til den omtalte trendanalyse der bør udvikles, og ville tage brug af de udviklede estimeringsmoduler.
Ved at have en sådan trendanalyse, giver det patienten mulighed for at blive opmærksom på ændring i adfærd, der kan være en nyttig ekstra hjælp til patienten.
Man skal dog være opmærksom på, at når sådanne produkter udvikles til brug inden for psykiatrien, har man med skrøbelige patienter at gøre.
Af den grund er det yderst vigtigt, at et sådant system ikke kan forårsage forværring for patienterne.
Det er af samme grund at systemet er blevet udviklet til at være oplysende fremfor dømmende, men systemet bør alligevel godkendes af relevante myndigheder inden det tages i brug af patienterne.

%Vurdering af perspektiver uden for psykiatrien (almen brug)
Systemet har også perspektiver uden for psykiatrien.
På nuværende stadie kan produktet også bruges til almen brug som en søvnestimerings eller fysisk aktivitetsestimerings applikation.
Man kan forestille sig at de udviklede estimeringsmoduler bruges i forskellige kontekster, hvor man vil på platformen knytte trendanalyse moduler på for patienterne, men hvor man kan nøjes med estimeringsmodulerne ved almen brug.
Løsningen ses altså ikke begrænset til psykiatrien men kunne også bruges til almen brug.

% Stort emne
% Fysisk aktivitet og søvn analyse kan lade sig gøre
% kræver mere arbejde
% Blive del af større software pakke (PsyLog)
% Ikke færdigt - men danner et grundlag for en mere komplet løsning
% Problemets natur gør at man aldrig er færdig - men har også været en tankegang der er fulgt i udviklingen (fleksibilitet - modularitet)
