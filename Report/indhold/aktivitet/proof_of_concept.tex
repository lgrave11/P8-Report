Metoden der bruges til at måle fysisk aktivitet blev valgt til at være skridt tælleren, da den tilnærmer sig fysisk aktivitet idet at hvis du bevæger dig virkelig meget vil din skridt tæller afspejle dette, og idet at vi tænker at fysisk aktivitet er en meget lavt hængende frugt vil vi ikke se på andre metoder.

\subsection{Aggregering}
For at informationen samlet af skridt tælleren kan være brugbar, skal den samles på en eller anden måden.
Her vælges det at samle data over hver dag, hvilket er fornuftigt under den antagelse at de patienter der kommer til at bruge det sover om natten, dvs fra ca 23 til 6-7.
Det er dog ikke sandsynligt at denne antagelse altid holder for folk med depression, da mange af dem sandsynligvis kunne finde på at ændre deres søvn tidspunkter som følge af depressionen.
Selvom antagelsen ikke holder, giver grupperingen af skridt i dage stadig fornuftigt billede af folks aktivitet, da brugere stadig er i stand til at se om de har bevæget sig mere eller mindre end de forrige dage og ud fra den information får et reelt grundlag at basere deres egen tilstands vurdering på.