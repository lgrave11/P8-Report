Denne sektion diskuterer det proof of concept der blev lavet til aktivitets delen af projektet. Samt diskutere den hvordan data aggregeres og hvilket videre arbejde der kunne være ved denne. 
Idéen ved at lave et proof of concept til fysisk aktivitet er for at vise at det er en meget lavt hængende frugt som nemt kan implementeres hvorefter vi forventer den kan arbejdes videre på, på et senere tidspunkt.

Metoden der bruges til at måle fysisk aktivitet blev valgt til at være skridt tælleren, da den tilnærmer sig fysisk aktivitet idet at hvis man bevæger sig meget vil ens skridttæller afspejle dette, og da vi tænker at fysisk aktivitet er en meget lavt hængende frugt vil vi ikke se på andre metoder.

Idet at skridt tælleren bruges er det eneste der skal gøres for at indsamle data er at lave et sensor modul som bruger den indebyggede skridttæller sensor i Android telefoner. 

\subsection{Aggregering}
For at informationen samlet af skridt tælleren kan være brugbar, skal den samles på en eller anden måden.
Her vælges det at samle data over hver dag, hvilket er fornuftigt under den antagelse at de patienter der kommer til at bruge det sover om natten, dvs fra ca 23 til 6-7.
Det er dog ikke sandsynligt at denne antagelse altid holder for folk med depression, da mange af dem sandsynligvis kunne finde på at ændre deres søvn tidspunkter som følge af depressionen.
Selvom antagelsen ikke holder, giver grupperingen af skridt i dage stadig fornuftigt billede af folks aktivitet, da brugere stadig er i stand til at se om de har bevæget sig mere eller mindre end de forrige dage og ud fra den information får et reelt grundlag at basere deres egen tilstands vurdering på.