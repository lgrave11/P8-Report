Da der ikke har været særlig meget fokus på fysisk aktivitet, er der god mulighed for videre arbejde med dette.

\subsection{Fra Skridt til Aktivitet}
Den nuværende funktionalitet til fysisk aktivitet er kun en skridttæller, fra denne ville det næste logiske skridt være at lave noget der kan give et overblik over hvor meget fysisk aktivitet der har været ud fra antallet af skridt.
Helt præcist hvordan dette skulle gøres og vises er uvist, men der er en mulighed for at det skulle kombineres med accelerometer data til at afgøre om de er tale om løb, gang eller anden form for bevægelse.
De forskellige gang typer skulle så også fortolkes som forskellige grad af aktivitet. 
Det ville også være nødvendigt at beslutte hvad for en enhed fysisk aktivitet er i, er det i Joule forbrændt eller noget andet?

Udover at beslutte hvordan fysisk aktivitet skal bestemmes skal man også beslutte sig for hvad det skal bruges til.
I sammenhæng med det primære fokus vi har på affektive lidelser, ville et oplagt fokus være at se på ændringer i mængden af fysisk aktivitet, da disse som tidligere nævnt (REF) \lars{Måske skal vi ref til fælles raport her}, er et godt signal for ændring i sindstilstand for patienter med affektive lidelser.

\subsection{Sammenhæng Mellem Søvn og Aktivitet}
Som nævnt i starten af dette kapitel, har vi en teori om, at der kunne være en sammenhæng mellem folks søvnkvalitet og deres bevægelsesmønster.
Denne teori burde undersøges, både i den forstand at man skulle undersøge om det er noget andre folk har overvejet, og undersøge om det er muligt at indsamle tilstrækkeligt data på en mobil enhed til at komme med noget brugbart omkring dette.

Denne teori bygger på, at vi finder det sandsynligt, at en dårlig søvn vil afspejles i måden man går på, da vi mener en person der har sover dårligt vil være mere tilbøjelig til for eksempel at slæbe sine fødder, end en veludhvilet person ville.
Vi er af den overbevisning at dette sagtens kunne være tilfældet, da vi i hvert fald selv føler dette er tilfældet.

\subsection{Andre Former for Aktivitets Målinger}
Indtil videre er der kun blevet lavet aktivitets udregninger v.h.a. skridttæller, men skridt er jo ikke den eneste form for fysisk aktivitet der kan være.
Andre former for fysisk aktivitet kunne for eksempel finde sted i et trænings center hvor der er tale om forskellige former for vægtløftning eller lignende.
Sådanne former for fysisk aktivitet bliver ikke dækket af en skridttæller og er formodentlig umulig at detektere via en telefon, for ikke at nævne at mange folk ikke har deres telefon på sig når de træner i et trænings center.
Derfor vil registrering af andre former for fysisk aktivitet kræve ekstra udstyr udover smartphone, sandsynligvis et smartwatch / wristband eller smart earplugs.
Her er earplugs nok den bedste mulighed, da det for mange ikke virker forstyrrende at have dem i ørerne, da de alligevel er vant til at høre musik mens de træner, og smart earplugs netop også er i stand til dette.

Der er dog en hvis sandsynlighed for at denne form for aktivitets registrering ikke er relevant for flertallet af patienter med affektive lidelser, da det kan være disse ikke går i træningscentre og træner, men dette vides ikke med sikkerhed.

De teknologier der kunne overvejes til fysisk aktivitet i et trænings center, kunne også give et bedre billede af fysisk aktivitet udenfor træningscenteret, da disse vil have adgang til ting som puls måling og galvanisk hud respons ved siden af det accelerometer data telefonen har til rådighed.
Derfor kunne tilføjelsen af suplerende datakilder muligvis give en bedre beskrivelse af fysisk aktivitet.