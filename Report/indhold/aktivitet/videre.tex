Denne sektion detaljerer forskellige ting som kunne arbejdes videre på ved aktivitet.
Idet at der ikke har været et stort fokus på fysisk aktivitet, er det idéelt at arbejde videre på.

\subsection{Fra Skridt til Aktivitet}
Den nuværende funktionalitet til fysisk aktivitet indeholder kun en skridttæller.
Eftersom skridttælleren ikke beskriver hvilken form for aktivitet man udøver vil dette være en nødvendighed at arbejde videre på.
Dette skyldes at man ikke præcist kan fortælle hvor meget fysisk aktivitet, der er blevet udøvet.
Hvordan man skal finde ud af hvilken form for aktivitet man laver ud fra skridttælleren er endnu uvist, men med en kombination af accelerometer-data vil dette muligvis være en mulighed.
Dog skal detaljegraden for den fysiske aktivitet også bestemmes, er det okay blot at se forskel på løb og gang, eller er det nødvendigt at se hvilken form for løb eller gangart der udføres.
Grunden til denne detaljegrad er vigtig, er fordi hvis man løber alt hvad man kan er det en anden form for aktivitet end hvis man løber sig en tur.

Ved at kunne detektere hvilken form for fysisk aktivitet en person udøver over en længere periode, vil det være muligt at danne et overblik over ændringer i fysisk aktivitet.
Denne ændring i fysisk aktivitet kan give et signal om ændring i sindstilstanden for en person.
Dette kan gøre sig klart gældende ved personer med affektive lidelser, som nævnt i \citet[Kapitel 1, Sektion 4]{misc:faellesrapp}, da en ændring i deres sindstilstand kan betyde at de er ved at få en depression- eller maniperiode.

\begin{comment}
Den nuværende funktionalitet til fysisk aktivitet er kun en skridttæller, fra denne vil det næste logiske skridt være at lave noget der kan give et overblik over hvor meget fysisk aktivitet der har været ud fra antallet af skridt.
Helt præcist hvordan dette skal gøres og vises er uvist, men der er en mulighed for at det skulle kombineres med accelerometer data til at afgøre om der er tale om løb, gang eller anden form for bevægelse.
De forskellige gang typer skal så også fortolkes som forskellige form for aktivitet. 
Det ville også være nødvendigt at beslutte hvad for en enhed fysisk aktivitet er i, er det i Joule forbrændt eller noget andet?

Udover at beslutte hvordan fysisk aktivitet skal bestemmes skal man også beslutte sig for hvad det skal bruges til.
I sammenhæng med det primære fokus vi har på affektive lidelser, vil et oplagt fokus være at se på ændringer i mængden af fysisk aktivitet, da det giver et signal om ændringer i sindstilstand for patienter med affektive lidelser, hvilket også blev nævnt i \citet[Kapitel 1, Sektion 4]{misc:faellesrapp}.
\end{comment}
\subsection{Sammenhæng Mellem Søvn og Aktivitet}
Som nævnt i starten af dette kapitel, har vi en teori om, at der kan være en sammenhæng mellem folks søvnkvalitet og deres bevægelsesmønster.
Teorien bør undersøges, både i den forstand at man skal undersøge om det er noget andre folk har overvejet, og undersøge om det er muligt at indsamle tilstrækkeligt data på en mobil enhed til at komme med noget brugbart omkring dette.

Derudover, bygger teorien på antagelsen, at dårlig søvn har en påvirkning på aktivitetsniveau.
For eksempel kunne en person, som har sovet dårligt, være mere tilbøjelig til at slæbe sine fødder end en veludhvilet person vil.
Vi er af den overbevisning at dette sagtens kan være tilfældet, men på samme tid vil det være en god idé at finde noget der underbygger det.

\subsection{Andre Former for Aktivitets Målinger}
Indtil videre er der kun blevet lavet aktivitetsudregninger ved hjælp af en skridttæller, idet at meget fysisk aktivitet vil tælle som skridt, såsom cykling og løb. 
Men der er andre former for fysisk aktivitet hvor det sandsynligvis ikke vil fungere specielt godt, for eksempel træning i et træningscenter hvor der er tale om forskellige former for vægtløftning. 
Det kan godt være at skridttælleren vil fungere delvist til dette, men den vil ikke kunne få hele billedet.

Sådanne former for fysisk aktivitet bliver ikke dækket af en skridttæller og er formodentlig umulig at detektere via en telefon, for ikke at nævne at mange personer muligvis ikke har deres telefon på sig når de træner i et træningscenter.
Derfor vil registrering af andre former for fysisk aktivitet kræve ekstra udstyr udover smartphone, sandsynligvis et smartwatch eller smartwristband.

Der er dog en hvis sandsynlighed for at denne form for aktivitetsregistrering ikke er relevant for flertallet af patienter med affektive lidelser, da det kan være disse ikke går i træningscentre, men dette vides ikke med sikkerhed.

De teknologier der kan overvejes til fysisk aktivitet i et træningscenter, kan også give et bedre billede af fysisk aktivitet udenfor træningscenteret, da disse vil have adgang til ting som puls måling og galvanisk hud respons ved siden af det accelerometer-data telefonen har til rådighed.
Derfor kunne tilføjelsen af supplerende datakilder muligvis give en bedre beskrivelse af fysisk aktivitet.