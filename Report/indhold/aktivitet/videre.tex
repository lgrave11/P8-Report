Da der ikke har været særlig meget fokus på fysisk aktivitet, er der god mulighed for videre arbejde med denne.

\subsection{Fra Skridt til Aktivitet}
Den nuværende funktionalitet til fysisk aktivitet er kun en skridttæller, fra denne ville det næste logiske skridt være at lave noget der kan give et overblik over hvor meget fysisk aktivitet der har været ud fra antallet af skridt.
Helt præcist hvordan dette skulle gøres og vises er uvist, men der er en mulighed for at det skulle kombineres med accelerometer data til at afgøre om de er tale om løb eller gang.
Det ville også være nødvendigt at beslutte hvad for en enhed fysisk aktivitet er i, er det i Joule forbrændt eller noget andet?

Udover at beslutte hvordan fysisk aktivitet skal bestemmes skal man også beslutte sig for hvad det skal bruges til.
I sammenhæng med det primære fokus vi har på affektive lidelser, ville et oplagt fokus være at se på ændringer i mængden af fysisk aktivitet, da disse som tidligere nævnt (REF TIL FÆLLESRAPPORT maybe?), er et godt signal for ændring i sindstilstand for patienter med affektive lidelser.

\subsection{Sammenhæng Mellem Søvn og Aktivitet}
Som nævnt i starten af dette kapitel, har vi en teori om, at der kunne være en sammenhæng mellem folks søvnkvalitet og deres bevægelsesmønster.
Denne teori burde undersøges, både i den forstand at man skulle undersøge om det er noget andre folk har overvejet, og undersøge om det er muligt at indsamle tilstrækkeligt data på en mobil enhed til at komme med noget brugbart omkring dette.

Denne teori bygger på, at vi finder det sandsynligt, at en dårlig søvn vil afspejles i måden man går på, da vi mener en person der har sover dårligt vil være mere tilbøjelig til for eksempel at slæbe sine fødder, end en veludhvilet person ville. 