Denne sektion beskriver forskningen, der undersøger forbindelsen mellem psykiske lidelser og fysisk aktivitet. 
Dette er vigtigt for at danne et overblik over hvad der overhovedet kan gøres med fysisk aktivitet, og til hvilke slags personer løsningen skal tilrettes.
Endvidere diskuteres der hvilke sensorer, der kan bruges til at måle fysisk aktivitet, såsom accelerometer, stepcounter og GPS.
Ydermere, diskuteres det proof of concept der implementeres for fysisk aktivitet og hvordan denne aggregeres. 

I vores samtaler med eksperter indenfor psykiatri feltet \citep[Sektion 1.3--1.4]{misc:faellesrapp}, blev det klart at de mente fysisk aktivitet var vigtigt i behandling af psykiske lidelser.
Det samme er sandt for fokusgruppeinterviewet, hvor vi igen blev informeret om at fysisk aktivitet hjælper \citep[Sektion 1.5]{misc:faellesrapp}.
Dette understøttes også af mange kilder, der har undersøgt en sammenhæng mellem fysisk aktivitet og symptomerne for depression. \citet{art:physDepSymptoms, Strawbridge15082002, Arredondo01072012} er blandt andet nået frem til at en sådan sammenhæng bestemt virker til at være reel.
Selvom kilderne ikke direkte siger at nedsat fysisk aktivitet er et symptom på begyndende depression, så er den sammenhæng de når frem til mellem depressions symptomerne og fysisk aktivitet nok til, at vi godt kan vurdere nedsat fysisk aktivitet som et symptom, da det gør tilstedeværelsen af andre symptomer mere alvorlige.

Derudover, har vi fundet kilder der også indikerer at det kan gå den anden vej, hvor forøget fysisk aktivitet er et tegn på bedring for depressive patienter.
Først har vi \citet{misc:healthReports} fra 1985, der siger at det ser ud til at være en sammenhæng, ikke blot mellem fysisk aktivitet og depression, men også med angst, mentalt handikap og andre former for psykiske lidelser.
Her var det muligt at finde nyere materiale, der understøtter forbindelsen mellem fysisk aktivitet og depression samt angst, hvilket er beskrevet herefter.

I \citet{art:physMental} fra 2000 er der blevet set på en større mængde undersøgelser, og de er nået frem til en række implikationer, der kan være for klinisk behandling.
Blandt disse implikationer er, at individer diagnosticeret med svær depression eller individer der har brug for psykologisk behandling, viser de største forbedringer ved øget fysisk aktivitet.
Derudover, er der også en implikation, der viser at fysisk aktivitet virker lige så effektivt som psykologisk terapi til at mindske effekten af depressions symptomer hos folk med mild eller moderat depression.

Ifølge \citet{book:sportPsyc} er der fundet en forbindelse mellem fysisk aktivitet og forbedring i tilstanden hos mange patientgrupper, heriblandt personer med angst, depression og stress.
Ydermere, præsenterer de også et eksperiment, hvor en række angst patienter testede effektiviteten af medicinsk behandling overfor en fysisk træningsplan og en kombination af disse.
Eksperimentet viste at selvom den medicinske behandling virkede bedre i starten, gav alle tre ca. samme resultat.
Efterfølgende viste det sig også, at de folk der havde været en del af det fysiske aktiveringsprogram, havde en væsentlig mindre risiko for tilbagefald.

Alle tre kilder sætter fokus på fysisk aktivitet som behandlingsmetode, \citet{art:physMental} og \citet{book:sportPsyc} har resultater der understøtter denne holdning, hvorimod \citet{misc:healthReports} kun har formodninger om at det virker.

Grundet disse resultater, virker det logisk at give folk mulighed for selv at overvåge deres egen fysiske aktivitet, hvilket for eksempel kunne gøres via deres smartphone, da der er mange sensorer i en smartphone der kan give information om dette.
Denne egen ansvarlige overvågning af fysisk aktivitet kan lægge dele af ansvaret for behandling og forhindring af tilbagefald over på patienten selv.
Ved at lægge ansvaret over på patienten og gøre dem opmærksomme på deres egen situation kan man forhåbentlig forhindre en del indlæggelser.
Dette skyldes at patienterne på et tidligere stadie kan søge læge, og dermed få startet behandlingen tidligere i forløbet.
Dette er i overensstemmelse med konceptet om patient empowerment som er beskrevet i \citep[Sektion 1.6.1]{misc:faellesrapp}. 

\subsection{Sensorer til Fysisk Aktivitet}
I en smartphone er der mange sensorer, som kan bruges til at få et indblik i den fysiske aktivitet hos smartphonens ejer.
Den primære sensor her er en stepcounter, der findes standard i alle android smartphones med API version 19 eller derover, men som også kan laves manuelt ved hjælp af accelerometret.
Udover stepcounteren, som giver et fint indblik i hvor mange skridt man tager, kan man også bruge GPS til at se på hvor langt man går.
Derudover, kan man også, for at være sikker på man logger den rigtige person, bruge smartphonens accelerometer data til at beregne gangarten \citep{4272626} og ud fra den vurdere om det er smartphonens ejer der har smartphonen, eller om det er en anden person der har lånt smartphonen.
Hvis det er en anden person der har smartphonen, bør denne aktivitet fjernes fra data der bruges for beregninger til fysisk aktivitet hos patienten.

Hvis man føler der er for stor risiko for at smartphonen ikke får det hele med, da det eksempelvis kan være irriterende at have sin smartphone med ude at løbe, kan man gøre brug af andre former for måleudstyr.
Eksempelvis kan man her bruge smart earplugs eller et smart wristband, der begge har en række indsamlingsmetoder, som kan bruges til at indsamle data af samme type som fra en smartphone.