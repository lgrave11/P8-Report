\section{Fysisk Aktivitet}
I vores samtaler med eksperter indenfor psykiatri feltet (ref til Jørgen og Janne), blev det klart at de mente fysisk aktivitet var vigtigt i behandling af psykiske lidelser.
Dette understøttes også af de kilder vi har fundet, først har vi \citet{misc:healthReports} fra 1985, der siger der ser ud til at være en sammenhæng ikke blot mellem depression og fysisk aktivitet, men også med angst, mentalt handikap og andre former for psykiske lidelser.
Her var det muligt at finde nyere materiale der understøtter forbindelsen med depression og angst, så man kan gå ud fra at de andre enten er blevet afvist eller ikke udforsket videre.

Ifølge \citet{book:sportPsyc} er der fundet en forbindelse mellem fysisk aktivitet og forbedring i tilstanden hos mange patientgrupper, heriblandt personer med angst, depression og stress.
Ydermere bliver der også på side 476 kolonne 1 i \citet{book:sportPsyc} beskrevet et eksperiment de har lavet, for de på en mængde angst patienter blev testet effektiviteten af medicinsk behandling overfor en fysisk træningsplan og en kombination af disse.
I det eksperiment nåede forskerne frem til, at selvom den medicinske behandling virkede bedre i starten, gav alle tre ca. samme resultat.
Ydermere viste det efterfølgende, at de folk der havde være en del af det fysiske aktiveringsprogram, havde en væsentlig mindre risiko for tilbagefald.
