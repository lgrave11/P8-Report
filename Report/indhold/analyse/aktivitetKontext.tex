\section{Fysisk Aktivitet}
I vores samtaler med eksperter indenfor psykiatri feltet (ref til Jørgen og Janne), blev det klart at de mente fysisk aktivitet var vigtigt i behandling af psykiske lidelser.
Dette understøttes også af de kilder vi har fundet, først har vi \citet{misc:healthReports} fra 1985, der siger der ser ud til at være en sammenhæng ikke blot mellem depression og fysisk aktivitet, men også med angst, mentalt handikap og andre former for psykiske lidelser.
Her var det muligt at finde nyere materiale der understøtter forbindelsen med depression og angst, så man kan gå ud fra at de andre enten er blevet afvist eller ikke udforsket videre.

I \citet{art:physMental} fra 2000 er der blevet set på en større mængde undersøgelser og de er nået frem til en række implikationer den information de finder kan have for klinisk behandling.
Blandt disse implikationer er, at individer diagnosticeret med svær depression eller individer der har brug for psykologisk behandling viser de største forbedringer ved øget fysisk aktivitet.
Derudover er der også en implikation at fysisk aktivitet virker lige så effektivt som psykologisk terapi til at mindske effekten af depressions symptomer hos folk med mild eller moderat depression.

Ifølge \citet{book:sportPsyc} er der fundet en forbindelse mellem fysisk aktivitet og forbedring i tilstanden hos mange patientgrupper, heriblandt personer med angst, depression og stress.
Ydermere bliver der også på side 476 kolonne 1 i \citet{book:sportPsyc} beskrevet et eksperiment de har lavet, for de på en mængde angst patienter blev testet effektiviteten af medicinsk behandling overfor en fysisk træningsplan og en kombination af disse.
I det eksperiment nåede forskerne frem til, at selvom den medicinske behandling virkede bedre i starten, gav alle tre ca. samme resultat.
Ydermere viste det efterfølgende, at de folk der havde være en del af det fysiske aktiveringsprogram, havde en væsentlig mindre risiko for tilbagefald.

Alle tre kilder sætter fokus på fysisk aktivitet som behandlingsmetode, \citet{art:physMental} og \citet{book:sportPsyc} har resultater der understøtter denne holdning hvorimod \citet{{misc:healthReports} kun har formodninger om at det virker.
Grundet disse resultater, virker det logisk at give folk mulighed for selv at overvåge deres egen fysiske aktivitet, hvilket fx kunne gøres via deres smartphone, da der er mange sensorer i en telefon der kan give information om dette.

\subsection{Sensorer til Fysisk Aktivitet}
I en smartphone er der mange sensorer der kan bruges til at få et indblik i den fysiske aktivitet hos telefonens ejer.
Den primære sensor her er en stepcounter, som der findes standard i alle android telefoner med version 19 eller derover.
Udover stepcounteren, som giver et fint indblik i hvor mange skridt man tager, kunne man også bruge GPS til at se på hvor stor en distance i bevægelse det så egentlig har givet, og ud fra dette måske også finde gennemsnits længden af ens skridt og hastigheden man har bevæget sig med, for at finde ud af hvor høj ens fysiske aktivitet er.
Derudover kan man også, for at være sikker på man har den rigtige person, bruge telefonens accellerometer data til at beregne gangarten og ud fra dette vurdere om det er telefonens ejer der har telefonen, eller om det er en anden person der har lånt den.

Hvis man føler der er for stor risiko for at telefonen ikke får det hele med, da det måske kunne være irriterende at have sin telefon med ud at løbe fx, kunne man gøre brug af andre former for måleudstyr.
Eksempelvis kunne man her bruge smart earplugs eller et smart wristband, der begge har en række indsamlingsmetoder, der giver data der kan bruges til samme resultater som data fra en smartphone.
