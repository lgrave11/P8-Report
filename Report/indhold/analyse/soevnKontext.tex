\section{Søvn}
Søvn er en vigtig del af alle personers hverdag, men både for meget og for lidt søvn kan være skadeligt og kan medføre mange problemer.

\subsection{Søvn Mangel}
Følgende afsnit er baseret på information fra \citep{misc:tenThings, misc:getSleep}

Søvn tillader visse dele af vores hjerne at hvile sig og lade op til næste dag, mangel på søvn forkorter disse hvileperioder, hvilket nedsætter funktionaliteten af visse dele af kroppen.
Den konsekvens de fleste er klar over er at man føler sig træt hvis man mangler søvn, mange vælger dog at dulme dette ved indtagelse of koffein i en eller anden form.
Indtagelsen af koffein løser dog ikke de problemer der er forbundet med søvnmangel. 

De konsekvenser søvnmangel kan have er bl.a.
\begin{itemize}
\item Dårligere dømmekraft
\item Glemsomhed
\item Forværret indlæringsproces
\item Dårligere reaktionstid
\item Overvægt
\item Hjertekar sygdomme
\item Diabetes
\item Nedsat immunforsvar
\end{itemize}

Alle disse problemer opstår, da kroppen og mere bestemt hjernen ikke når at genoplade nok hvis man ikke får nok søvn.
Mens man sover slapper de dele af hjernen, der påvirker de nævnte områder, af og kroppen bruger også denne tid til at reparere beskadigede celler og lade dem op igen så de er klar til den nye dag.
Hvis der ikke er nok tid til denne proces vil det resultere i nedsat funktionalitet i de områder der er påvirket.
I tilfælde af immunforsvaret, bliver meget af kroppens energi omdirigeret hertil når man sover og bruges til at genopbygge de celler man måske har mistet i løbet af dagen.
Her vil søvnmangel så betyde at immunforsvaret har færre ressourcer til rådighed, hvilket kan ses i at folk med søvnmangel generelt set er længere om at komme sig efter sygdom end folk med normalt søvnmønster.

Ud fra dette kan det ses at søvnproblemer har en række alvorlige konsekvenser, dog er de fleste af dem først noget der fremkommer efter længerevarende søvnmangel.
Dette gør, at overvågning af eget søvnmønster er relevant, også for folk der ikke har depression, da en lang række er disse konsekvenser er farlige for individet.
Der er endda også fare for andre involveret i form af den dårligere reaktionstid, da dette kombineret med kørsel af bil eller lignende, er særdeles farligt for andre.
Det er ikke kun i trafikken søvnmangel giver problemer, der er også eksempler på folk med søvnmangel har påvirket væsentlig værre ulykker end et trafikuheld, som eksempel på dette nævner \citet{misc:tenThings} bl.a. den radioaktive nedsmeltning af Tjernobyl i 1986.

%effekt af for meget
\subsection{For Meget Søvn}
%fundne sources
%http://www.webmd.com/sleep-disorders/excessive-sleepiness-10/10-results-sleep-loss
%http://healthysleep.med.harvard.edu/need-sleep/whats-in-it-for-you/health
%http://www.healthline.com/health/sleep-deprivation/effects-on-body
%http://serendip.brynmawr.edu/exchange/node/1690
Dette afsnit er baseret på information fra \citep{misc:oversleep, misc:longsleepsurvey}

Det at sove for meget kan have lignende helbredsmæssige problemer som mangel på søvn, beskrevet ovenfor.
De risikoer der nævnes er diabetes, overvægt, hovedpine, ryg smerter, hjerteproblemer.
\citet{misc:oversleep} oplyser også at der er forskere der har fundet en forbindelse mellem at sove for meget og dødsfald.
Der er også en forbindelse mellem for meget søvn og depression, dette er dog som oftest at meget eller kort søvn er et symptom på depression.

Udover de ovennævnte risici ved for meget søvn, nævner \citet{misc:brainsleep} at der er fundet en forbindelse mellem for meget søvn og øget fald i kognitive funktioner hos folk i 60erne og 70erne.

