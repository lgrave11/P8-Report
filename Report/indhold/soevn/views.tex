\subsection{Visualisering}\label{sec:soevnVisVidArb}
Visualisering af analyse resultater er kun blevet udforsket i begrænset omfang, selvom den visuelle repræsentation er yderst essentiel i afgørelsen om systemet kan bruges, da denne er fundet til kriteriet om at brugerne selv skal kunne bruge det uden hjælp, beskrevet i \cref{sec:metodevalg}.
Hvis visningsmoduler skal implementeres på et tilfredsstillende niveau, kræver det at der bliver udforsket datavisualiserings-teknikker og -teorier.
Det vigtigste her ville så være at finde en visualiserings-form, der giver den information brugeren leder efter eller har brug for, på en sådan måde at brugere selv kan forstå hvad der vises på skærmen.
En anden ting der er vigtig når visningsmoduler skal laves er, at det ikke er nok med en, da folk med forskellig baggrund forstår ting forskelligt, derved kan det være at den visualisering der kan være indlysende for en person, kan være fuldstændig uforståelig for en anden.
Det problem er en af hovedårsagerne til, at visualisering har fået så lidt opmærksomhed som det har, det er simpelthen et projekt i sig selv at lave noget, der er forståeligt for så bred en gruppe folk som dette system er tiltænkt til.
Dog er der lavet et proof of concept af visningsmoduler, der er beskrevet i \cref{sec:pocVis}.