\subsection{Sensor Vægtning}
Som nævnt i \cref{subsec:kombimodeller} vægtes de forskellige sensorer baseret på vægtninger i \citet{6563918}, hvilket gøres idet at kilden siger deres vægtninger er baseret på læring fra deres eksperimenter, dog med ændringer idet vi bruger færre sensor kilder.
Dette blev valgt af simplificerende årsager både for udviklere og brugere, da disse faste værdier gør at en lærings periode ikke er nødvendig, og at systemet derfor kan bruges direkte.

At vægtningen er statisk gør også systemet til en mere generel løsning, hvilket indenfor psykiatrien kan være problematisk, da alle individer anses som værende forskellige. 
Eksempelvis er nogle personer mere støjende end andre, også om dagen, hvorfor at ved en stille person kunne det være man skal give søvnestimeringen baseret på lyd en mindre vægt end for en støjende person der er stille når han sover.

Hvis vi ville have vores sensor vægtning og dermed vores søvnestimering til at være mere tilpasset individet, ville det være nødvendigt at tilføje en lærings periode med data der kan anses for den objektive sandhed.
Dette træningsdata kunne eksempelvis indsamles ved at lade patienten holde en dagbog for deres søvn samtidig med at man indsamler sensor.
Denne indsamling ville så køre i en periode, og modellen ville så kunne tilpasses den enkelte patient.

Baseret på dette ville det så være muligt at lave en model til at finde den bedste vægtning af datakilder.
Den bedste vægtning værende den hvor nøjagtigheden af vurderingerne er så høj som muligt.

En model implementering ville også kræve længere testperioder end dem vi har brugt.
De nuværende testperioder vi har brugt, har været på en til to dage, hvorimod med en model ville en længere og sandsynligvis todelt testperiode være nødvendig.
Perioden skal være længere da der er brug for mere data til at lave en ordentlig model med procent vægtning, og den bliver bør være todelt så man har noget data man kan teste sin endelige model på som ikke har været brugt til at lave modellen.
