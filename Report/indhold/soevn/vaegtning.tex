\subsection{Sensor Vægtning}
Som nævnt er vægtningen mellem vores sensorer sat til statiske værdier.
Dette blev valgt af simplificerende årsager både for udviklere og brugere, da disse faste værdier gør at en læringsperiode ikke er nødvendig, og at systemet derfor kan bruges direkte.

At vægtningen er statisk kan være problematisk, da alle individer anses som værende forskellige. 
Eksempelvis er nogle personer mere støjende end andre, også om dagen, hvorfor at ved en stille person kunne det være man skal give søvnestimeringen baseret på lyd en mindre vægt, end for en støjende person der er stille når han sover.

Hvis vi ville have vores sensor vægtning, og dermed vores søvnestimering, til at være mere tilpasset individet, vil det være nødvendigt at tilføje en læringsperiode med data, der kan anses for den absolutte sandhed.
Dette træningsdata kan eksempelvis indsamles ved at lade patienten holde en dagbog for deres søvn, samtidig med at man indsamler sensor data.
Denne indsamling vil så køre i en periode, og modellen vil så kunne tilpasses den enkelte patient.

Baseret på dette vil det så være muligt at lave en model til at finde den bedste vægtning af datakilder.
Den bedste vægtning værende den hvor nøjagtigheden af vurderingerne er så høj som muligt.

Implementeringen vil også kræve længere testperioder end dem vi har brugt.
De nuværende testperioder vi har brugt, har været på en til to dage, hvorimod med en læringsmodel vil en længere og sandsynligvis todelt testperiode være nødvendig.
Perioden skal være længere da der er brug for mere data til at lave en ordentlig læringsmodel med procent vægtning, og den bør være todelt så man har noget data man kan teste sin endelige model på, som ikke har været brugt til at lave modellen.
