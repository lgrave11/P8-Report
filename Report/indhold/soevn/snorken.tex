\subsection{Snorken}
Snorken er et åbenlyst problem når man skal opdage søvn ved hjælp af sensorer, idet at en af de primære sensorer der bruges til at bestemme om personen sover er lyd amplitude.
Hvis denne lyd amplitude gentagne gange optager snorken vil sandsynlighedens udregningen for denne sensor falde hver eneste gang snorken bliver optaget, og dette er helt klart et problem som skal arbejdes videre på.

Udfra følgende artikler dannes der et grundlag for hvordan snorken registreres \citep{Dafna2013} og \citep{Calabrese20111101}.

I \citep{Dafna2013} brugte forskere et system til at inddele snorken og andre akustiske hændelse gennem et søvnforløb ved brug af lyd data der blev optaget med en mikrofon ved en polysomnigrafisk undersøgelse. 
De endte med et meget præcist system der kunne adskille snorken og andre akustiske hændelser med ~98\% nøjagtighed.
Denne kunne tilpasses ind i vores system på den måde at hvis snorken kan opdages og filtreres fra har den ikke nogen indflydelse på sandsynligheds beregningen af søvn. 
De bruger lyd data og ikke bare amplituden, så det vil være nødvendigt at undersøge om systemet også kunne fungere ved bare amplituden.

I \citep{Calabrese20111101} forslår forskere et system der skal bruges til diagnosticering af søvnapnø ved hjælp af optaget lyd data, men de implementerede kun en prototype af systemet og havde ikke evalueret systemet ordentligt. 