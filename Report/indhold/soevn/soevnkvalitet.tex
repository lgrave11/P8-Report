\subsection{Søvnkvalitet}
Ved fokusgruppemødet, se \citep[Kapitel 1, Sektion 5]{misc:faellesrapp}, blev der fortalt at et at problemerne patienter ved mødet personligt oplevede var at de generelt sover dårligt under en depression.
Dette skyldes at deres søvn bliver meget afbrudt, at de kan have det svært ved at falde i søvn og at de ikke sover særlig længe.
Fra andre, se \citep[Kapitel 1, Sektion 3 og 4]{misc:faellesrapp}, har vi hørt at et symptom personer med affektive lidelser oplever tit er søvnforandringer. 

Det der for det meste er blevet fokuseret på er at måle søvnlængde, og idet ser vi ikke på aktuelle søvnkvalitets indikatorer som f.eks. urolighed, hvor lang tid det tager dem i falde i søvn og hvor afbrudt søvnen er.
For dette projekt er det blevet vist at søvnlængde er en af de bedste indikatorer og hvis man kunne finde en måde at måle dette kan man med fordel få en meget god idé om hvordan det går med patienter ved at se på forandringer i søvnmønster. 
Hvis man vil have det fulde billede af søvn, skal man også se på søvnkvalitet da at hvis en person vågner hele tiden men ikke larmer eller rører sin telefon bliver dette registreret som en lang søvn periode.