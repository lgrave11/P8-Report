\subsection{Søvnkvalitet}
Ved fokusgruppeinterviewet, se \citep[Sektion 1.5]{misc:faellesrapp}, blev der fortalt at et af problemerne som patienterne ved interviewet personligt oplevede var, at de generelt sover dårligt under en depression.
Endvidere har en psykiater samt en psykolog fortalt os at personer med affektive lidelser ofte oplever en søvnforandring, se \citep[Sektion 1.3--1.4]{misc:faellesrapp}.
Eksempel på dette kan være at deres søvn er meget afbrudt, at de har det svært ved at falde i søvn og at de ikke sover særlig længe.

Det der for det meste er blevet fokuseret på er at måle søvnlængde. 
Der ses dermed ikke på aktuelle søvnkvalitetsindikatorer som f.eks. urolighed, hvor lang tid det tager dem at falde i søvn og hvor afbrudt søvnen er.
Søvnlængde er en god indikator og hvis man kan finde en måde at måle dette, kan der fås en idé om hvordan det går med patienten, ved blot at se på forandringer i søvnmønstret. 
Hvis man dog vil have det fulde billede af søvn, skal man også se på søvnkvalitet. 
Dette skyldes at selvom man har en lang søvnperiode kan den stadig være af ringe kvalitet.