\subsection{Søvnkvalitet}
Ved fokusgruppemødet, se \citep[Kapitel 1, Sektion 5]{misc:faellesrapp}, blev der fortalt at et af problemerne patienter ved mødet personligt oplevede var at de generelt sover dårligt under en depression.
Endvidere har en psykiater samt en psykolog at personer med affektive lidelser ofte oplever en søvnforandring, se \citep[Kapitel 1, Sektion 3 og 4]{misc:faellesrapp}.
Eksempel på dette kan være at deres søvn er meget afbrudt, at de har det svært ved at falde i søvn og at de ikke sover særlig længe.

Det der for det meste er blevet fokuseret på er at måle søvnlængde, og dermed ses der ikke på aktuelle søvnkvalitets indikatorer som f.eks. urolighed, hvor lang tid det tager dem at falde i søvn og hvor afbrudt søvnen er.
Søvnlængde er en god indikator og hvis man kunne finde en måde at måle dette kan man med fordel få en meget god idé om hvordan det går med patienter ved at se på forandringer i søvnmønster. 
Hvis man dog vil have det fulde billede af søvn, skal man også se på søvnkvalitet da at hvis en person vågner hele tiden men ikke larmer eller rører sin telefon bliver dette registreret som en lang søvn periode.