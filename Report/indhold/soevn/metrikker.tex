For at kunne tale om indsamlet data på en fornuftig vis, gives her en beskrivelse af de metrikker der benyttes.
Metrikkerne der omtales stammer fra udviklede data indsamlingsmoduler fra \citet{misc:faellesrapp}.

Acceleration måles i $\frac{m}{s^2}$ med et interval på $215-230 ms$.
Det vil sige at acceleration måles som ændring i hastighed per tid.

Amplituden måles som maks amplituden over en tidsperiode af $1000ms$.
Enheden er en arbitrær enhed der varierer fra smartphone til smartphone, men svarer til en skala fra $0-100\%$ elektrisk spænding, der kan måles på den indbyggede mikrofon, konverteret til en 16 bits integer værdi.

Lys måles i Lux i et interval på $215-230 ms$.

Lås/låst op måles per begivenhedsbasis og registeres med et tidspunkt for når handlingen skete, samt en boolsk værdi for om der blev låst eller låst op.
Det samme gælder for opladning/stoppet opladning, samt tændt/slukket.

For at få et bedre overblik over de omtalte metrikker se \cref{tab:metrikker}.

\begin{table}[h]
\begin{tabular}{c?c|c}
	 Datatype & Enhed & Interval \\ 
	\thickhline Acceleration & $\frac{m}{s^2}$ & $215-230ms$ \\ 
	\hline Amplitude & Afhængig af mikrofon & $1000ms$ \\ 
	\hline Lysstyrke & Lux & $215-230ms$ \\ 
	\hline Låst/låst op & Bolsk værdi & På begivenhedsbasis \\ 
	\hline Opladning/ikke opladning & Bolsk værdi & På begivenhedsbasis \\ 
	\hline Tændt/slukket & Bolsk værdi & På begivenhedsbasis \\ 
	
\end{tabular}
\caption{Overblik over metrikker}\label{tab:metrikker} 
\end{table}