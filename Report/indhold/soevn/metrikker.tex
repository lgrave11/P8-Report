For at kunne tale om indsamlet data på en fornuftig vis gives her en beskrivelse af metrikkerne der benyttes.

Acceleration måles i $\frac{m}{s^2}$ med et interval på $215-230 ms$.
Dvs. der måles acceleration der er ændring i hastighed per tid.

Amplituden måles som max amplituden over en tidsperiode af $1000ms$ varighed.
Enheden er en arbitrær enhed der varierer fra telefon til telefon, men svarer til en skala fra $0-100\%$ elektrisk spænding der kan måles på den indbyggede telefon, skaleret op til en 16 bits integer værdi.

Lys måles i Lux i et interval på $215-230 ms$.

Lås/låst op måles per begivenheds basis og registeres med et timestamp for når handlingen skete samt en boolsk værdi for om der blev låst eller låst op.
Det samme gælder for opladning/stoppet opladning, samt tændt/slukket.

For at få et bedre overblik over de omtalte metrikker se \cref{tab:metrikker}.

\begin{table}[h]
\begin{tabular}{|c|c|c|}
	\hline Datatype & Enhed & Interval \\ 
	\hline Acceleration & $\frac{m}{s^2}$ & $215-230ms$ \\ 
	\hline Amplitude & Afhængig af mikrofon & $1000ms$ \\ 
	\hline Lysstyrke & Lux & $215-230ms$ \\ 
	\hline Låst/låst op & Bolsk værdi & På begivenhedsbasis \\ 
	\hline Opladning/ikke opladning & Bolsk værdi & På begivenhedsbasis \\ 
	\hline Tændt/slukket & Bolsk værdi & På begivenhedsbasis \\ 
	\hline 
\end{tabular}
\caption{Overblik over metrikker}\label{tab:metrikker} 
\end{table}