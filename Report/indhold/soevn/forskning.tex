Denne sektion beskriver hvad der vides akademisk om søvn og forbindelsen til psykiske lidelser, idet denne slags information er meget vigtig hvis man vil vide hvad der er produktivt at arbejde med og hvilke slags personer løsningen skal være tilrettet. 
Sektionen diskuterer videre hvilke slags indsamlede data der kan benyttes. 
Dette er relevant da det er godt at vide når man skal analysere videre på de metrikker. 
Endvidere diskuteres forskellige metoder til søvnestimeringsmetoder da disse skal sammenlignes og diskuteres videre.

Det viser sig at en regelmæssig dags cyklus er særdeles vigtig for bipolare patienter.
Eksempelvis kan mangel på søvn medføre større risiko for at gå ind i en depressions- eller maniperiode \citep{CPSP:CPSP1164}.

Hvis man er i en maniperiode sover man væsentligt mindre end for den habituelle periode \citep{CPSP:CPSP1164}.
Ved depression er det sædvanligt at sove væsentligt længere end for den habituelle periode, men vi har også hørt fra patienter at de har sovet meget lidt under en depression \citep[Kapitel 2, Sektion 5]{misc:faellesrapp}.

Grundet dette anses søvn for en vigtig indikator på om man begynder at befinde sig i en depressions- eller maniperiode og bør undersøges nærmere til at determinere sådanne perioder.

Desuden har \citet{art:sleepCusMania} udført et eksperiment på en enkelt testperson der indikerede at søvnmangel kunne føre til maniperioder som så oftest blev efterfulgt af en depression når personen havde fået lidt søvn.
Der er dog et par problemer med denne kilde, det første værende at den kun bruger en enkelt test person så det kunne være det bare var et særtilfælde og ikke et generelt træk.
Det andet problem er at informationen er fra 1991, så den kunne godt være viden om forbindelsen mellem søvn og mani har ændret sig siden, men den er blevet citeret flere gange siden, men denne information understøttes af nyere artikler som f.eks. \citet{barbini1996sleep} der viser et tydeligt sammenhæng fra søvn til mani, men også fra mani til søvn.

Underbyggende er søvn vigtig for mange andre ting end bare sindstilstand.
Eksempelvis indikeres der at søvnmangel også kan føre til hjerteproblemer \citep{Mullington2009294,art:sleeplossHeart}, men også at det kan svække ens immunforsvar \citep{misc:sleepImmune}.

Derved kan vi sige at søvn er en vigtig faktor for forebyggelse af sygdomme og forøget livskvalitet, og derfor er der lagt en stor fokus på forskning i dette område.
Som resultat af dette fokus er der en lang række søvnestimeringsmetoder, hvor de mest fremtrædende muligheder nævnes og vurderes som inspiration til vores løsning.
For at have en ordentlig grundlag for at diskutere datakilder er de benyttede metrikker dog først beskrevet.

\subsection{Metoder}
Med et fælles grundlag for dataindsamlingsmetrikker følger hermed en beskrivelse af en række søvnestimeringsmetoder.

\subsubsection{Polysomnografi}\label{sec:polysomnografi}
Den måske mest akkurate søvnestimeringsmetode er polysomnografi, der er en metode der kombinerer et elektroencefalogram med målinger af muskel spænding og øjenbevægelse \citep{misc:polysomnografi,misc:polysomnography}.
Metoden kræver en stor mængde special udstyr og teknikere med specialiseret viden og erfaring til at montere udstyr og fortolke data.
Derfor er denne metode upraktisk at udføre i en patients soveværelse.

\subsubsection{AktiGrafi}
Imidlertid findes der andre metoder der er nemmere at benytte.
Et eksempel på sådan en søvnestimeringsmetode er aktigrafi, der er akkurat selvom det kun benytter sig af accelerometre påmonteret ens arm.
En sådan teknik kan estimere metrikker såsom timer sovet, søvn virkningsgrad, og antal af søvnafbrydelser.
Kendte eksempler på sådanne apparater findes ved eksempelvis FitBit og JawBone \citep{misc:fitbitSleepTracking,misc:jawBoneSleepTracking}.
Prisen på disse kan svinge fra 500 kroner til 2000 kroner baseret på hvilket produkt man køber, og kræver at armbåndet er påmonteret når man sover.

Kravet om at udstyr som FitBit og JawBone skal være monteret på ens arm finder vi ikke som en tilstrækkelig hæmning for at afvise brugen af sådan en teknik.
Af samme grund står muligheden åben for at bruge sådant udstyr i fremtiden, hvis man er interesseret i mere akkurat søvnestimering end efterfølgende nævnte estimeringsmetoder.
Derudover er platformen opbygget til at være yderst fleksibel angående hvilke moduler kan benyttes, så skulle der ønskes at udvikle et modul der virker med FitBit og JawBone, er dette muligt.

\subsubsection{Søvn Dagbog}
I stedet for en objektiv løsning, kan man benytte patientens subjektive viden om deres søvn ved f.eks. at bruge en søvn dagbog, hvori patienten svarer på en række spørgsmål om søvnen. 
På den måde får patienten ansvaret for at svare på spørgsmål om deres søvn, hvilket gør at man kan følge søvnrytmens udvikling.
Endvidere, er dette også værd at tage i betragtning, når man skal lære en ny model. 
Et eksempel på dette kan ses i \citet{Min:2014:TNT:2556288.2557220}, hvor dagbogen bruges som en "ground truth" til at træne ens model.
\begin{comment}
Med ressourcerne for dette projekt er dette dog ikke valgt som primær fokus, da vi ønsker en metode der kan estimere søvn med mindst mulig bruger intervention, og hvor grundlaget for vurdering skal være ved hjælp af sensorer fremfor en subjektiv vurdering man alligevel har mulighed for i forvejen.
Idéen om en "objektiv dagbog" er dermed også i tankerne her, og er hvorfor denne løsning ikke undersøges nærmere end at det kan bruges for læringsperioder for vores modeller.
\end{comment}

\subsubsection{Toss 'N' Turn}\label{sec:tossNturn}
Følgende fremgangsmåde er præsenteret i \citet{Min:2014:TNT:2556288.2557220}, og beskrivelsen bygger på forsknings resultaterne derfra.

For Toss 'N' Turn fremgangsmåden er tanken at man blot skal have sin smartphone i sit soveværelse for at den kan estimere ens søvnstarttidspunkt, vækketidspunkt og søvnlængde.
Teknikken tager udgangspunkt i en række sensorkilder der er tilgængelig på smartphonen i forvejen.
Disse værende accelerometer, mikrofon(maksamplitude), lyssensor, proximitysensor, kørende processer, batteristadie (opladning/ikke opladning), og skærmvisningstilstand.
Ud fra disse sensorkilder og en søvndagbog der foretages i minimum tre dage til at lære en søvnestimeringsmodel, kan de opnå en præcision med søvnlængde Mean Error (ME) på under 1 time.
Deres metode har en præcision for daglig søvnestimering på 92\% mens deres søvnkvalitetsestimering har en præcision på 84\%, hvilket gør det til en oplagt mulighed at arbejde med.
Metoden fungerer så ved at foretage en række feature udtrækninger, og så bruge teknikker såsom exponential moving average, Naive Bayes classifier og decision trees til at opnå den nøjagtighed.
Feature utrækninger er basalt set hvor man udfra et dataset bygger deriverede værdier der har til hensigt at være informative og kan bruges til efterfølgende læring.

Denne metode kræver en oplæringsperiode på minimum 3 dage for at opnå maksimal præcision.
Derudover pålægger metoden ikke at man skal have sin smartphone placeret i sengen, hvorfor denne metode stiller knap så store krav til patienten.

\subsubsection{Best Effort Sleep Model}\label{sec:BES}
Best Effort Sleep (BES) Model \citet{6563918} bruger en fremgangsmåde der udelukkende baseres på målbare data fra en smartphone til at estimere en brugers søvnlængde, meget lig Toss 'N' Turns koncept.
Disse målbare datakilder anskaffes på en måde så brugeren ikke behøver ændre sin søvnadfærd, som det for eksempel er tilfældet på andre tilsvarende løsninger som for eksempel JawBone eller FitBit der kræver at en wearable benyttes mens man sover.
Det at ingen af datakilderne er i direkte kontakt med brugeren går dog ud over præcisionen, da der er op mod 40 minutters unøjagtighed i estimeringen af søvnlængde, hvilket for en person der sover 8 timer svarer til 92\% nøjagtighed.
Ud over dette kan BES ikke bruges til at estimere kvaliteten af ens søvn.
Med kvaliteten menes hvor mange gange man har været vågen i løbet af natten, til dette kræves involvering af brugeren, enten at de bruger en wearable eller at de interagerer med systemet.

Fremgangsmåden fungerer godt som en backup løsning til en af de brugerinvolverende fremgangsmåder, da det der kendertegn dem er af de ikke vil give pålidelige resultater hvis patienten ikke bruger dem korrekt.
I de tilfælde hvor BES og en brugerinvolverende løsning begge er tilstede, kan BES bruges til at validere data fra den brugerinvolverende løsning. 

I BES bruges der seks forskellige datakilder til at estimere søvn.
Grunden til det store antal datakilder er at ingen af dem er i direkte kontakt med brugeren og at de hver for sig ikke siger særlig meget men i kombination giver en udmærket estimering om brugeren sover.
Kilderne der bruges er meget tilsvarende dem der bruges i Toss 'N' Turn beskrevet i \cref{sec:tossNturn}, hvor der her bruges lys, lyd, bevægelse (accelerometer), skærmlås, batteri status og om den er slukket.
Målinger fra disse datakilder skal vægtes forskelligt, og ud fra eksperimenterne udført af \citet{6563918}, nåede de frem til følgende vægtninger der kan ses i \cref{tab:vaegtninger}.

\begin{table}[h]
\centering
\begin{tabular}{|c|c|}
\hline Datakilde & Koefficient\\
\hline Lys & 0.0415 \\ 
\hline Lås & 0.0512 \\ 
\hline Slukket & 0.0000 \\ 
\hline Opladning & 0.0469 \\ 
\hline Bevægelse & 0.5445 \\ 
\hline Lyd & 0.3484 \\ 
\hline 
\end{tabular}
\caption{De forskellige vægtninger for hver datakilde.}
\label{tab:vaegtninger}
\end{table}

Disse koefficienter kan bruges af os til at springe en eventuel lærings periode over, dog kan dette gå ud over præcisionen. 
Da den omtalte algoritme ikke er offentlig tilgængelig kan det være et problem at stole på de ovenstående koefficienter, da en efterfølgende udvikling af en algoritme højst sandsynlig ikke er helt tilsvarende den af \citet{6563918}.
Dog hvis denne tilgang bruges kan man som start bruge de nævnte koefficienter og så efterfølgende tilpasse koefficienterne så de passer til ens specifikke implementation af algoritmen.
Hvis præcisionen er meget vigtig vil man have en læringsperiode, hvor man får den objektive sandhed om søvnlængden ind, så man kan blive i stand til at finde de optimale koefficient værdier for hver kilde.
Endvidere vil analysen også bedre kunne tilpasses individet, frem for at bruge koefficienterne i \cref{tab:vaegtninger}, der vil gøre analysen mere generel.

Fordelen ved BES er at den ikke kræver nogen form for brugerinteraktion, udover at ens smartphone skal ligge tæt på hvor man sover.
Derudover er en fordel også at denne model er klar til at blive taget i brug med det samme hvis vi vælger at holde os til de koefficienter der blev præsenteret ovenover.
Metoden kræver at patienten lægger sin smartphone i nærheden af sengen f.eks. på sit natbord.

\subsubsection{Statistisk Baseret Tilgang}\label{sec:statbased}
Den statiske baserede tilgang bliver præsenteret af \citet{misc:statbased}.
Denne metode bygger på at telefonen ligger i patientens seng og er ud fra de bevægelser der sker, i stand til at bestemme om patienten sover eller ej. 
Metoden ser på  accelerationsmålinger i sæt af fire minutter, og bestemmer for hver periode om patienten sover eller ej.
Til at nå frem til dette ser metoden på intensiteten af bevægelser for hver måling og sammenligner dette med en beregnet tærskelværdi.
Hvis en defineret procent af målingerne overstiger den beregnede tærskelværdi, vurderes patienten til at være vågen, ellers estimeres det at patienten sover.
Denne procent kan justeres og kunne læres v.h.a. machine learning.
Kilden oplyser at metoden har en præcision på 68\%, hvilket er relativt lavt i forhold til førnævnte metoder.

\subsection{Metrikker}
For at kunne tale om indsamlet data på en fornuftig vis, gives her en beskrivelse af de metrikker der benyttes.
Metrikkerne der omtales stammer fra udviklede data indsamlingsmoduler fra \citet{misc:faellesrapp}.

Acceleration måles i $\frac{m}{s^2}$ med et interval på $215-230 ms$.
Det vil sige at acceleration måles som ændring i hastighed per tid.

Amplituden måles som maks amplituden over en tidsperiode af $1000ms$.
Enheden er en arbitrær enhed der varierer fra telefon til telefon, men svarer til en skala fra $0-100\%$ elektrisk spænding, der kan måles på den indbyggede mikrofon, konverteret til en 16 bits integer værdi.

Lys måles i Lux i et interval på $215-230 ms$.

Lås/låst op måles per begivenhedsbasis og registeres med et tidspunkt for når handlingen skete, samt en boolsk værdi for om der blev låst eller låst op.
Det samme gælder for opladning/stoppet opladning, samt tændt/slukket.

For at få et bedre overblik over de omtalte metrikker se \cref{tab:metrikker}.

\begin{table}[h]
\begin{tabular}{c?c|c}
	 Datatype & Enhed & Interval \\ 
	\thickhline Acceleration & $\frac{m}{s^2}$ & $215-230ms$ \\ 
	\hline Amplitude & Afhængig af mikrofon & $1000ms$ \\ 
	\hline Lysstyrke & Lux & $215-230ms$ \\ 
	\hline Låst/låst op & Bolsk værdi & På begivenhedsbasis \\ 
	\hline Opladning/ikke opladning & Bolsk værdi & På begivenhedsbasis \\ 
	\hline Tændt/slukket & Bolsk værdi & På begivenhedsbasis \\ 
	
\end{tabular}
\caption{Overblik over metrikker}\label{tab:metrikker} 
\end{table}