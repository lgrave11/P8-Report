\begin{comment}
Dette kapitel indeholder forskning omkring søvn, hvor forskellige metoder til at detektere søvn bliver beskrevet.
Hvorefter metoderne bliver sammenlignet med hinanden for at se på hvilket metoder vil kunne bruges til dette projekt.
%Et eller andet omkring konfigurationstabel
Til slut vil der blive set på en enkelt metode i dybden som et proof of concept.
\end{comment}
Søvn og IT er måske ikke den mest indlysende kombination for mange folk, men der er alligevel mulighed for at lave denne kombination.
Søvn og IT i sammenhæng er blevet muligt med den nye teknologi, der er begyndt at komme frem som for eksempel smartphones, smartwatches og smartwristbands.
Der er dog i denne sammenhæng mange ting man skal tage højde for, blandt disse er privatlivs regler om overvågning og andre lignende regler om privatlivets fred.

Problemet er så at bruge den data man har til rådighed til at finde ud af hvornår en person sover.
Når man gør dette er det vigtigt man beslutter sig for hvilke kriterier man vægter højest.
Eksempler på kriterier kunne f.eks. være,

\begin{itemize}
	\item Præcision - At udregninger i høj grad passer med virkeligheden.
	\item Ikke-påtrængende - At udregninger kræver minimal bruger interaktion.
	\item Pålidelig - Systemet skal kunne være fleksibelt i hvilke situationer og på hvilken måde det kan bruges og stadig give fornuftige resultater.
\end{itemize}


Disse kriterier kan godt være modstridende, for eksempel vil prioritering af et ikke-påtrængende system sandsynligvis komme på med en bekostning af præcision, da det at gøre systemet ikke-påtrængende gør det sværere at analysere på den data man har.
For at tilgodese ønsket om et ikke-påtrængende system, kunne man for eksempel opstille det krav at systemet skulle være uafhængig af bruger input.
Kriteriet om pålidelighed kunne opfyldes ved at stille det krav til systemet at det for eksempel skulle være i stand til at filtrere data fra der ikke er fra den person det er meningen skal bruge systemet.

Der vælges først at se på hvilken forskning der allerede er blevet lavet i søvn estimerings området.
På baggrund af den information dannes der et billede af søvn delen af projektet, der udformer sig som en konfigurationstabel.
Derefter ses der på eksisterende søvn estimerings metoder og disse bruges som inspiration til vores egen løsning.
Til sidst præsenteres vores løsning og hvad muligheder for videre arbejde der er med den løsning. 



