\begin{comment}
Dette kapitel indeholder forskning omkring søvn, hvor forskellige metoder til at detektere søvn bliver beskrevet.
Hvorefter metoderne bliver sammenlignet med hinanden for at se på hvilket metoder vil kunne bruges til dette projekt.
%Et eller andet omkring konfigurationstabel
Til slut vil der blive set på en enkelt metode i dybden som et proof of concept.
\end{comment}
Søvn og IT er ikke den mest indlysende kombination for mange folk, men der er dog alligevel mulighed for at lave denne kombination.
Søvn og IT i sammenhæng er blevet muligt med alt det nye teknologi der er begyndt at komme frem som for eksempel smartphones, smartwatches, smartwristbands og etc.
Der er dog i denne sammenhæng mange ting man skal tage højde, blandt disse er privatlivs regler om overvågning og andre lignende regler lavet til at beskytte den individuelle person.

Problemet her bliver så at klassificere hvornår en person sover ud fra den data man har til rådighed.
Når man arbejder med klassificering af noget som helst, er det vigtigt man beslutter sig for hvilke kriterier man vægter højest.
Kriterier til dette kan være præcision, simpelt, "non-intrusive" eller pålidelighed.
Med præcision menes om man vægter korrekte resultater højt, og her ville mange nok sige at selvfølgelig vægter man præcision højt, da man ikke kan bruge det til noget hvis en situation klassificeres forkert, men her er spørgsmålet mere hvilke kriterier man vægter højere end præcision.
Med det menes hvilke kriterier man er villig til at sænke præcisionen for at opfylde bedre.
Pålidelighed er ikke det samme som præcision, med pålidelighed menes at systemet skal kunne være mere fleksibelt i hvilke situationer det kan bruges og på hvilke måder det kan bruges, og stadig give fornuftige resultater.
Simpelt betyder at processen bag skal være nem at forstå og sætte sig ind i.
Non-intrusive betyder at personer der bruger system ikke skal ændre på deres adfærd for at bruge det, hvilket vil sige at introduktionen af systemet til problemområdet ikke ændrer på problemområdet.

Ikke alle disse kriterier går i helt samme retning, for eksempel vil prioritering af et non-intrusive system sandsynligvis komme på en bekostning af præcisionen, da det at gøre systemet non-intrusive sætter en begrænsning på hvilke former for data man har til rådighed.
For at tilgodese ønsket om et non-intrusive system, kunne man for eksempel opstille det krav at systemet skulle være uafhængig af bruger input, dog kan dette blive problematisk i et klassificerings problem som dette er i sidste ende.
Kriteriet om pålidelighed kunne opfyldes ved at stille det krav til systemet at det for eksempel skulle være i stand til at filtrere data fra der ikke er fra den person det er meningen skal bruge systemet.

\lars{Lav secrefs}
Med alt denne information bag en søvn analyse vælges der først at se på hvilket forskning der allerede er blevet lavet på området med søvnbestemmelse.
Ud fra den information vi får fra denne undersøgelse laver vi en konfigurationstabel til at beskrive hvordan vi forstiller os systemet vil udforme sig.
Derefter vælger vi hvilke af de fundne metoder vi tager inspiration i til vores egen løsning som vi så følger om med en beskrivelse af hvordan vi tog den information i brug.
Til sidst præsenteres hvad vi finder frem til med systemet og hvad der er mulighed for af videre arbejde med vores løsning. 