Det er blevet mere tilgængeligt at analysere søvn med ny teknologi, som for eksempel smartphones, smartwatches og smartwristbands.
Der er dog i denne sammenhæng mange ting man skal tage højde for, blandt disse er privatlivsregler om overvågning og andre lignende regler om privatlivets fred.

Problemet er så at bruge data man har til rådighed, til at finde ud af hvornår en person sover.
Når man gør dette er det vigtigt man beslutter sig for hvilke kriterier man vægter højest.
Eksempler på kriterier kunne f.eks. være,

\begin{itemize}
	\item Præcision - At udregninger i høj grad passer med virkeligheden.
	\item Ikke-påtrængende - At udregninger kræver minimal bruger interaktion.
	\item Pålidelig - Systemet skal kunne være fleksibelt i hvilke situationer og på hvilken måde det kan bruges og stadig give fornuftige resultater.
\end{itemize}

Disse kriterier kan godt være modstridende, for eksempel vil prioritering af et ikke-påtrængende system sandsynligvis komme på bekostning af præcision, da det at gøre systemet ikke-påtrængende begrænser den typen af data man har tilrådighed.
For at tilgodese ønsket om et ikke-påtrængende system, kunne man for eksempel opstille det krav at systemet skal være uafhængig af bruger input.
Kriteriet om pålidelighed kan opfyldes ved at stille det krav til systemet, at det for eksempel skal være i stand til at filtrere data fra der ikke er fra den person som systemet skal måle på.

Der vælges først at se på hvilken forskning, der allerede er blevet lavet i søvnestimerings-området.
På baggrund af den information dannes der et billede af søvn delen af projektet, der udformer sig som en konfigurationstabel.
Derefter ses der på eksisterende søvnestimeringsmetoder og disse bruges som inspiration til vores egen løsning.
Til sidst præsenteres vores løsning og hvad muligheder for videre arbejde der er med den løsning. 