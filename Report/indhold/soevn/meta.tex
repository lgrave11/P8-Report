Før vi kan tale om søvn er der nødt til at blive opstillet en fælles forståelse af hvad søvn og vågenhed indebærer.
\begin{description}
		\item[Søvn] er en tilstand af hvile. Dette er karakteriseret af fuldt eller delvis tab af bevidsthed, så der er en nedsættelse af kropslig bevægelse og respons til stimulus. Formålet med søvn er blandt andet at få kroppens kræfter genoprettet \citep{misc:SleepDefinition}.
		\item[Vågenhed] er den modsatte tilstand af søvn, hvor individet er bevidst.
\end{description}

Det er blevet mere tilgængeligt at analysere søvn med ny teknologi, som for eksempel smartphones, smartwatches og smartwristbands.
Der er dog i denne sammenhæng mange ting man skal tage højde for, blandt disse er privatlivsregler om overvågning.

Problemet er så at bruge data man har til rådighed, til at finde ud af hvornår en person sover.
Når man gør dette er det vigtigt man beslutter sig for hvilke kriterier man vægter højest.
Eksempler på kriterier kunne f.eks. være,

\begin{itemize}
	\item Præcision - At udregninger i høj grad passer med virkeligheden.
	\item Ikke-påtrængende - At udregninger og dataindsamlingen ikke påvirker patientens dagligdag.
	\item Pålidelig - Systemet skal fungere i en lang række kontekster og stadig give fornuftige resultater.
\end{itemize}

Disse kriterier kan godt være modstridende, for eksempel vil prioritering af et ikke-påtrængende system sandsynligvis komme på bekostning af præcision, da det at gøre systemet ikke-påtrængende begrænser typen af data man har til rådighed.
For at tilgodese ønsket om et ikke-påtrængende system, kunne man for eksempel opstille det krav at systemet skal være uafhængig af brugerinput.
Kriteriet om pålidelighed kan opfyldes ved at stille det krav til systemet, at det for eksempel skal være i stand til at filtrere data fra, der ikke er fra den person som systemet skal måle på.

Der vælges først at se på hvilken forskning, der allerede er blevet lavet i søvn\-es\-ti\-me\-rings\-om\-rå\-det.
På baggrund af den information dannes der et billede af søvn delen af projektet, der udformer sig som en konfigurationstabel.
Derefter ses der på eksisterende søvnestimeringsmetoder og disse bruges som inspiration til vores egen løsning.