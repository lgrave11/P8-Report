Det er vigtigt at have en fælles forståelse for begreberne der benyttes i denne rapport.
Hvis ikke man har en sådan fælles forståelse risikeres det at man taler forbi hinanden.

Af denne grund følger herefter en gennemgang af vigtige begreber der benyttes i denne rapport.

\begin{description}[style=nextline]
	\item[Søvn] Søvn er en tilstand af hvile. Dette er karakteriseret af fuldt eller delvis tab af bevidsthed, så der er en nedsættelse af kropslig bevægelse og respons til stimulus. Formålet med søvn er delvist at få kroppens kræfter genoprettet \citep{misc:SleepDefinition}.
	\item[Vågenhed] Vågenhed er den modsatte tilstand af søvn, hvor individet er bevidst.
	\item[Fysisk aktivitet] Fysisk aktivitet er en tilstand af kropslig bevægelse der kræver mere energi end inaktivitet \citep{misc:PhysicalActivity}.
	\item[Fysisk inaktivitet] Fysisk inaktivitet er den tilstand hvor kropslig bevægelse er minimal, modsat fysisk aktivitet.
\end{description}
