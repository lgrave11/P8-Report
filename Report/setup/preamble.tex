%  A simple AAU report template.
%  2015-05-08 v. 1.2.0
%  Copyright 2010-2015 by Jesper Kjær Nielsen <jkn@es.aau.dk>
%
%  This is free software: you can redistribute it and/or modify
%  it under the terms of the GNU General Public License as published by
%  the Free Software Foundation, either version 3 of the License, or
%  (at your option) any later version.
%
%  This is distributed in the hope that it will be useful,
%  but WITHOUT ANY WARRANTY; without even the implied warranty of
%  MERCHANTABILITY or FITNESS FOR A PARTICULAR PURPOSE.  See the
%  GNU General Public License for more details.
%
%  You can find the GNU General Public License at <http://www.gnu.org/licenses/>.
%
\documentclass[11pt,twoside,a4paper,openright]{report}
%%%%%%%%%%%%%%%%%%%%%%%%%%%%%%%%%%%%%%%%%%%%%%%%
% Language, Encoding and Fonts
% http://en.wikibooks.org/wiki/LaTeX/Internationalization
%%%%%%%%%%%%%%%%%%%%%%%%%%%%%%%%%%%%%%%%%%%%%%%%
% Select encoding of your inputs. Depends on
% your operating system and its default input
% encoding. Typically, you should use
%   Linux  : utf8 (most modern Linux distributions)
%            latin1 
%   Windows: ansinew
%            latin1 (works in most cases)
%   Mac    : applemac
% Notice that you can manually change the input
% encoding of your files by selecting "save as"
% an select the desired input encoding. 
\usepackage[utf8]{inputenc}			% Character encoding
%\usepackage[utf8]{inputenc}
% Make latex understand and use the typographic
% rules of the language used in the document.
\usepackage[danish]{babel}

\usepackage{tabu}
\usepackage{array}
\newcolumntype{?}{!{\vrule width 1.1pt}}
\usepackage{makecell}
% Use the palatino font
\usepackage[sc]{mathpazo}
\newcommand{\thickhline}{\Xhline{1.1pt}}
\linespread{1.05}         % Palatino needs more leading (space between lines)
% Choose the font encoding
%\usepackage[T1]{fontenc}
%%%%%%%%%%%%%%%%%%%%%%%%%%%%%%%%%%%%%%%%%%%%%%%%
% Graphics and Tables
% http://en.wikibooks.org/wiki/LaTeX/Importing_Graphics
% http://en.wikibooks.org/wiki/LaTeX/Tables
% http://en.wikibooks.org/wiki/LaTeX/Colors
%%%%%%%%%%%%%%%%%%%%%%%%%%%%%%%%%%%%%%%%%%%%%%%%
% load a colour package
\usepackage{xcolor}
\definecolor{aaublue}{RGB}{33,26,82}% dark blue
% The standard graphics inclusion package
\usepackage{graphicx}
\graphicspath{{./grafik/}}
\usepackage{rotating}
% Set up how figure and table captions are displayed
\usepackage{caption}
\captionsetup{%
  font=footnotesize,% set font size to footnotesize
  labelfont=bf % bold label (e.g., Figure 3.2) font
}
\usepackage{pdfpages}
\usepackage[natbib=true, style=numeric, backend=bibtex, sorting=none,maxbibnames=99]{biblatex}
\DefineBibliographyStrings{danish}{andothers = {et\addabbrvspace al\adddot}}
\setcounter{biburlucpenalty}{8000}
\setcounter{biburllcpenalty}{7000}
% Make the standard latex tables look so much better
\usepackage{array,booktabs}
% Enable the use of frames around, e.g., theorems
% The framed package is used in the example environment
\usepackage{framed}
\usepackage[linewidth=2pt]{mdframed} %Bliver brugt til at lave en ramme om ting

%%%%%%%%%%%%%%%%%%%%%%%%%%%%%%%%%%%%%%%%%%%%%%%%
% Mathematics
% http://en.wikibooks.org/wiki/LaTeX/Mathematics
%%%%%%%%%%%%%%%%%%%%%%%%%%%%%%%%%%%%%%%%%%%%%%%%
% Defines new environments such as equation,
% align and split 
\usepackage{amsmath}
% Adds new math symbols
\usepackage{amssymb}
% Use theorems in your document
% The ntheorem package is also used for the example environment
% When using thmmarks, amsmath must be an option as well. Otherwise \eqref doesn't work anymore.
\usepackage[framed,amsmath,thmmarks]{ntheorem}

%%%%%%%%%%%%%%%%%%%%%%%%%%%%%%%%%%%%%%%%%%%%%%%%
% Page Layout
% http://en.wikibooks.org/wiki/LaTeX/Page_Layout
%%%%%%%%%%%%%%%%%%%%%%%%%%%%%%%%%%%%%%%%%%%%%%%%
% Change margins, papersize, etc of the document
\usepackage[
  inner=28mm,% left margin on an odd page
  outer=41mm,% right margin on an odd page
  ]{geometry}
% Modify how \chapter, \section, etc. look
% The titlesec package is very configureable
\usepackage{titlesec}
\titleformat{\chapter}[display]{\normalfont\huge\bfseries}{\chaptertitlename\ \thechapter}{20pt}{\Huge}
\titleformat*{\section}{\normalfont\Large\bfseries}
\titleformat*{\subsection}{\normalfont\large\bfseries}
\titleformat*{\subsubsection}{\normalfont\normalsize\bfseries}
%\titleformat*{\paragraph}{\normalfont\normalsize\bfseries}
%\titleformat*{\subparagraph}{\normalfont\normalsize\bfseries}

% Clear empty pages between chapters
\let\origdoublepage\cleardoublepage
\newcommand{\clearemptydoublepage}{%
  \clearpage
  {\pagestyle{empty}\origdoublepage}%
}
\let\cleardoublepage\clearemptydoublepage

% Change the headers and footers
\usepackage{fancyhdr}
\pagestyle{fancy}
\fancyhf{} %delete everything
\renewcommand{\headrulewidth}{0pt} %remove the horizontal line in the header
\fancyhead[RE]{\small\nouppercase\leftmark} %even page - chapter title
\fancyhead[LO]{\small\nouppercase\rightmark} %uneven page - section title
\fancyhead[LE,RO]{\thepage} %page number on all pages
% Do not stretch the content of a page. Instead,
% insert white space at the bottom of the page
\raggedbottom
% Enable arithmetics with length. Useful when
% typesetting the layout.
\usepackage{calc}

%%%%%%%%%%%%%%%%%%%%%%%%%%%%%%%%%%%%%%%%%%%%%%%%
% Bibliography
% http://en.wikibooks.org/wiki/LaTeX/Bibliography_Management
%%%%%%%%%%%%%%%%%%%%%%%%%%%%%%%%%%%%%%%%%%%%%%%%
%\usepackage[backend=bibtex,
%  bibencoding=utf8
%  ]{biblatex}
%\addbibresource{bib}

%%%%%%%%%%%%%%%%%%%%%%%%%%%%%%%%%%%%%%%%%%%%%%%%
% Misc
%%%%%%%%%%%%%%%%%%%%%%%%%%%%%%%%%%%%%%%%%%%%%%%%
% Add bibliography and index to the table of
% contents
\usepackage[nottoc]{tocbibind}
% Add the command \pageref{LastPage} which refers to the
% page number of the last page
\usepackage{lastpage}
% Add todo notes in the margin of the document
\usepackage[
%  disable, %turn off todonotes
  colorinlistoftodos, %enable a coloured square in the list of todos
  textwidth=\marginparwidth, %set the width of the todonotes
  textsize=scriptsize, %size of the text in the todonotes
  ]{todonotes}

%%%%%%%%%%%%%%%%%%%%%%%%%%%%%%%%%%%%%%%%%%%%%%%%
% Hyperlinks
% http://en.wikibooks.org/wiki/LaTeX/Hyperlinks
%%%%%%%%%%%%%%%%%%%%%%%%%%%%%%%%%%%%%%%%%%%%%%%%
% Enable hyperlinks and insert info into the pdf
% file. Hypperref should be loaded as one of the 
% last packages
\usepackage{hyperref}
\hypersetup{%
	pdfpagelabels=true,%
	plainpages=false,%
	pdfauthor={Author(s)},%
	pdftitle={Title},%
	pdfsubject={Subject},%
	bookmarksnumbered=true,%
	colorlinks=true,%
	citecolor=black,%
	filecolor=black,%
	linkcolor=black,% you should probably change this to black before printing
	urlcolor=black,%
	pdfstartview=FitH%
}

\usepackage{enumitem}
\usepackage{caption}
\usepackage{subcaption}
\usepackage[danish,nameinlink,capitalise]{cleveref}
\usepackage{listings}
\usepackage{verbatim} % For the comment environment, that allows for commenting out large blocks.

% Additional commands
\newcommand{\namedtodo}[5]
{
  \ifthenelse{\equal{#1}{}}
  {
    \todo[backgroundcolor=#4,caption=
    {\textbf{#3: } #2}
    ,inline]
    {\color{#5}\textbf{#3: }#2}
  }
  {
    \todo[backgroundcolor=#4,caption=
    {\textbf{#3: } #1}
    ,inline]
    {\color{#5}\textbf{#3: }#2}
  }
}
\newcommand{\mikkel}[2][]{\namedtodo{#1}{#2}{Mikkel}{blue!80}{white}}
\newcommand{\stefan}[2][]{\namedtodo{#1}{#2}{Stefan M}{orange}{black}}
\newcommand{\mikael}[2][]{\namedtodo{#1}{#2}{Mikael}{green}{black}}
\newcommand{\bruno}[2][]{\namedtodo{#1}{#2}{Bruno}{black!10!red!90}{white}}
\newcommand{\lasse}[2][]{\namedtodo{#1}{#2}{Lasse}{black!10!yellow!90}{black}}
\newcommand{\als}[2][]{\namedtodo{#1}{#2}{Als}{purple!90!orange}{white}}
\newcommand{\winde}[2][]{\namedtodo{#1}{#2}{Winde}{black}{white}}
\newcommand{\lars}[2][]{\namedtodo{#1}{#2}{Lars}{blue!80}{black}}
\newcommand{\ivan}[2][]{\namedtodo{#1}{#2}{Ivan}{red}{black}}


  \makeatletter \renewcommand \listoftodos{\section*{List of Todos} \@starttoc{tdo}}
  \renewcommand\l@todo[2]
    {\par\noindent \textit{#2}, \parbox{10cm}{#1}\par} \makeatother
% Her er en liste over navnene på de forskellige styles
% C#: csharp
% F#: fsharp

% 
% Listings kan refereres vha. \cref{}
\crefname{listing}{code example}{code example}
\Crefname{listing}{Code example}{code examples}
% 

%Algoritmer i cref
\crefname{algocf}{algorithm}{algorithm}
\Crefname{algocf}{Algorithm}{Algorithms}
%Algoritmelinjer i cref
\crefalias{AlgoLine}{line}%

\makeatletter
\let\cref@old@stepcounter\stepcounter
\def\stepcounter#1{%
  \cref@old@stepcounter{#1}%
  \cref@constructprefix{#1}{\cref@result}%
  \@ifundefined{cref@#1@alias}%
    {\def\@tempa{#1}}%
    {\def\@tempa{\csname cref@#1@alias\endcsname}}%
  \protected@edef\cref@currentlabel{%
    [\@tempa][\arabic{#1}][\cref@result]%
    \csname p@#1\endcsname\csname the#1\endcsname}}
\makeatother
%

% Angivelse af navn på listings
\renewcommand\lstlistingname{Code example}
\renewcommand\lstlistlistingname{Code example}

\lstdefinestyle{standard}
{
	frame=shadowbox,
	framesep=5pt,
	rulecolor=\color{blue!40!black},
	rulesepcolor=\color{white!93!black},
	numbers=left,
	basicstyle=\ttfamily,
	numberstyle=\tiny,
	numberfirstline=true,
	%numberblanklines=false,
	stepnumber=1,
	numbersep=9pt,	
	captionpos=b,
	escapeinside={(*}{*)},
	breaklines=true,
	tabsize=4,
	language=c
}

\lstset{style=standard}

\lstdefinestyle{c}
{
	style=standard
}

\lstdefinestyle{csmall}
{
	style=c
}

\lstdefinestyle{csharp}
{
	style=standard,
	language=[Sharp]C
}
\lstdefinestyle{csharpsmall}
{
	style=csharp
}
\lstdefinestyle{fsharp}
{
	language=[Sharp]F,
	frame=lr,
	rulecolor=\color{blue!80!black}
}
\lstdefinestyle{fsharpsmall}
{
	style=fsharp,
	basicstyle=\ttfamily\footnotesize
}


% Definitions

% Superscript and subscript
\newcommand{\superscript}[1]{\ensuremath{^{\textrm{#1}}}}
\newcommand{\subscript}[1]{\ensuremath{_{\textrm{#1}}}}

% Degrees
\newcommand{\degree}{\ensuremath{^\circ}}
\newcommand{\dg}{\degree}

\newcommand{\quoter}[1]%
{
  \par
  \vspace{1.5em}
  \addtolength{\leftskip}{1.5cm}
  \addtolength{\rightskip}{1.5cm}
  \textit{#1}
  \addtolength{\leftskip}{-1.5cm}
  \addtolength{\rightskip}{-1.5cm}
  \vspace{1.5em}
  \par
}


\newcommand{\sensor}[3]
{
	\section{#1}
	#2
	
	#3
}

\newcommand{\analyse}[2]
{
\subsection{#1}
#2
}


